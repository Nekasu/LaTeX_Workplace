\vspace {5mm}
\centerline
{
  \zihao{5}
  \begin{CJK*}{UTF8}{zhhei}参~考~文~献\end{CJK*}
}

\begin{thebibliography}{99}
\zihao{5-} \addtolength{\itemsep}{-1em}
\vspace {1.5mm}

\bibitem[1]{1}
网上的文献(举例:The Cooperative
Association for Internet Data Analysis(CAIDA),http://www.caida.org/data
2010,7,18) \textbf{*请采用脚注放于正文出现处,每页的脚注从1开始编序号*}\footnote{The Cooperative Association for Internet Data
Analysis (CAIDA), http://www.caida.org/data 2010, 7, 18}

\bibitem[2]{2} 中文的参考文献需给出中英文对照。形式如[3]。

\bibitem[3]{3} Zhou Yong-Bin, Feng Deng-Guo. Design and analysis of cryptographic
protocols for RFID. Chinese Journal of Computers, 2006, 29(4): 581-589 (in
Chinese) \newline
(周永彬, 冯登国. RFID安全协议的设计与分析. 计算机学报, 2006, 29(4): 581-589)

\bibitem[4]{4} 期刊、会议、书籍名称不能用缩写。

\bibitem[5]{5} 作者(外国人姓在前,名在后可缩写, 后同).
题目(英文题目第一字母大写,其它均小写):副标题(如果有). 刊名(全称), 年,
卷(期): 页码 \textbf{*期刊论文格式*}

\bibitem[6]{6}作者.
文章题目(英文题目第1字母大写,其它均小写):副标题(如果有)//Proceedings of
the {\ldots} (会议名称). 会议召开城市, 会议召开城市所在国家, 年: 页码
\textbf{*会议论文集论文格式*}

\bibitem[7]{7}作者. 文章题目(英文题目第一字母大写, 其它均小写):
副标题(如果有)//编者. 文集标题. 出版地: 出版社, 出版年: 页码
\textbf{*文集格式*}

\bibitem[8]{8}作者. 书名: 副标题(如果有). 版次(初版不写). 出版社地点: 出版社,
出版年 \textbf{*书籍格式*}

\bibitem[9]{9}作者. 文章题目[博士学位论文/硕士学位论文]. 单位名称,单位地点, 年
\textbf{*学位论文格式*}

\bibitem[10]{10}作者. 文章题目(英文题目第一字母大写,其它均小写). 单位地点: 单位,
技术报告: 报告编号, 年 \textbf{*技术报告*}

\bibitem[11]{11}专利拥有人. 专利名称,专利授权国家,专利授权日期
\textbf{*技术专利*}
  \end{thebibliography}

% \begin{strip}
% \end{strip}