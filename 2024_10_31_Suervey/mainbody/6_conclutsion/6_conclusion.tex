\section{Conclusion}

Since its rise in 2016 [2], neural style transfer has seen significant growth. Historically, style transfer has evolved from slow to real-time in terms of speed, from arbitrary style transfer to specific style transfer, and then to arbitrary transfer in terms of style realization, and from simple to complex in terms of structure. Differences in researchers' understanding of the concept of style have led to significant differences in implementation methods. However, current approaches to improving style transfer mainly fall into three categories: enhancing the quality of style transfer, optimizing the runtime, and increasing the interactivity of style transfer. Of these, interactivity is relatively less studied.

Meanwhile, as style transfer has developed, researchers in other fields have also adopted the ideas of style transfer. Some researchers have focused on applying style transfer in everyday life (e.g., [5,15]); others have applied it in the field of artistic design (e.g., [10-14]); still others have applied style transfer methods to other research areas, such as adversarial example research [8,9,17,18], image generation [19], domain adaptation [20], font generation [6], and font recognition [7].

It seems that style transfer is no longer a field for mere self-amusement but has become a research direction that can provide new perspectives for other areas. Although the field of style transfer has made significant progress, challenges remain. These challenges will continue to drive researchers to develop deeper, propose more advanced models and algorithms, and further develop style transfer technology. In the future, research in style transfer may focus more on integrating with other technologies, such as combining self-supervised learning and multimodal learning, to open up more innovative application scenarios.
