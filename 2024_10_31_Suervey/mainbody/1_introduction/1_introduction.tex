\section{introduction}

A picture is worth a thousand words and excellent artworks frequently provide information distinct from that of real photographs. However, without long-term professional training, an ordinary person may find it difficult to independently create a piece of art that satisfies themselves or others. Moreover, the time and cost required to train a true artist are often immeasurable. Even for a skilled artist proficient in creating stylized works, completing a single art piece requires a amount of time. To efficiently convert real photos into artistic images, the task of image style transfer has emerged.

Image style transfer aims to combine the content of a real photograph with the style of an artwork, creating a stylized image that simultaneously embodies the content of the photograph and the style of the artwork. In style transfer tasks, the image providing the content is referred to as the content image, while the image providing the style is called the style image. The generated result is known as the stylized image. This paper primarily focuses on image style transfer; unless otherwise specified, the term "style transfer" in the following text refers specifically to image style transfer.

The development of style transfer can be divided into two stages. The first stage spans from the mid-1990s\citep{01jing2019neural} to 2016\citep{02gatys2016image}, characterized by the use of mathematical models for texture simulation. The second stage, from 2016\citep{02gatys2016image} to the present, is marked by the use of deep learning and neural networks for style transfer. The former is relatively traditional, while the latter incorporates new methods. Therefore, this paper refers to the first stage as "traditional style transfer" and the second stage as "neural style transfer."

Traditional style transfer typically employs mathematical and signal processing techniques, such as texture synthesis, histogram matching, and filtering. These methods involve manipulating pixels to simulate the desired style. For example, frequency domain filtering can be used to enhance or suppress certain frequency components of an image, thereby altering its appearance. The advantage of traditional style transfer lies in its higher computational speed and lower resource consumption, but it may struggle to capture more advanced artistic styles and textures.

In contrast, neural network-based style transfer methods are more flexible and efficient. Deep learning techniques are used to learn and apply image styles. They train neural networks to capture the features of different artistic styles, which are then applied to the input image to generate a new image with the desired style. The strength of neural style transfer lies in its ability to better capture the details and complexities of artistic styles, though the required time and resource consumption can vary significantly depending on the network structure.

Traditional style transfer and neural style transfer should not be viewed as mutually exclusive. On the contrary, some of the recent neural style transfer advancements\citep{03li2023frequency,04huang2017arbitrary} draw inspiration from traditional style transfer and digital image processing. Meanwhile, neural network-based style transfer techniques also have certain drawbacks, such as artifacts and difficulties in controlling the stylization process. Combining traditional and neural style transfer methods could potentially yield better results.

Image style transfer has a wide range of practical applications, such as in environmental atmosphere rendering\citep{05ke2023neural}, font generation\citep{06fu2023neural}, font recognition\citep{07tang2022few}, portrait editing \citep{08liu2021psgan++,09xu2022transeditor}, design assistance \citep{10liu2021self,11bae2023unsupervised,12hollein2022stylemesh,13yin20213dstylenet,14yang2022industrial}, photo restoration\citep{15gunawan2023modernizing}, virtual reality (VR), and augmented reality (AR)\citep{16mu20223d}.
Furthermore, as a fundamental task in computer vision, style transfer can assist in other research areas, such as adversarial example\citep{17naseer2022stylized,18cao2023stylefool}, image generation\citep{19karras2019style}, and domain adaptation\citep{20guan2022cdtnet}. Whether from the perspective of practical applications or scientific research, the task of style transfer has broad applications.

This paper primarily introduces the task of image style transfer, and its structure is organized as follows. Firstly, it introduces the achievements of traditional style transfer. Secondly, it covers the achievements of the neural style transfer. Thirdly, it presents the evaluation parameters in the style transfer field and compares representative achievements in the field. Fourthly, it discusses the application of style transfer in other fields and non-image style transfer tasks. Finally, it discusses the unresolved issues in the field of style transfer. The main contributions of this paper are as follows:
\begin{enumerate}
    \item Introducing some achievements in style transfer in the chronological order and providing a subdivision method for neural style transfer.
    \item Comparing some representative achievements in the field of style transfer.
    \item Providing a comprehensive summary and analysis of objective evaluation metrics in the field of style transfer. There is considerable debate over the choice of objective metrics in style transfer, with different studies employing a wide variety of metrics that vary significantly. To the best of our knowledge, this is the first work to systematically summarize and analyze the objective evaluation metrics used in the majority of recent style transfer studies.
    \item Discussing the existing issues in the field of style transfer.
\end{enumerate}