\section{Processing and Analysis of Data}

\subsection{Data Preprocessing}
Data preprocessing is a crucial step in agricultural remote sensing data analysis, involving data correction and data fusion. Data correction ensures the accuracy and consistency of remote sensing data, while data fusion integrates multi-source data to enhance the spatial, temporal, and spectral resolution, providing richer information for subsequent analysis.

Data Correction\cite{huangAgriculturalRemoteSensing2018}.
Data correction involves processing raw remote sensing data to eliminate various errors, ensuring data authenticity and accuracy. Common types of data correction include atmospheric correction, geometric correction, and radiometric correction.

Atmospheric Correction\cite{seelanRemoteSensingApplications2003}. 
Atmospheric correction involves eliminating the effects of the atmosphere on remote sensing data. Due to the scattering and absorption by atmospheric molecules and aerosols, the electromagnetic signals received by remote sensing sensors are distorted and attenuated. Therefore, atmospheric correction is a key step in accurately restoring surface reflectance. Here are some methods to do this task.
Absolute Correction Method: Based on atmospheric radiation transfer models (such as MODTRAN and 6S), correction is performed by retrieving atmospheric parameters (such as water vapor content and aerosol optical thickness).
Relative Correction Method: Utilizing ground truth data or concurrent multispectral images for relative correction, such as the radiometric normalization method and dark object subtraction method.

Geometric Correction\cite{geRemoteSensingSoil2011}.
Geometric correction maps remote sensing images to a geographic coordinate system, eliminating geometric distortions caused by sensor attitude, terrain relief, and Earth's curvature. Here are some methods to do this task.Geometric Precision Correction: Utilizing ground control points (GCP) and high-precision digital elevation models (DEM), correction is achieved through polynomial transformation or rigorous orthorectification. Orthorectification: Using DEM to eliminate terrain-induced image distortions, aligning images with their true ground positions.

Radiometric Correction\cite{ulloAdvancesIoTSmart2021}.
Radiometric correction converts raw image DN values into actual surface reflectance or radiance values, eliminating the influence of sensor performance and observation conditions on radiometric values.Here are some methods to do this task.
Absolute Radiometric Correction: Using sensor calibration parameters and solar irradiance for correction to calculate surface reflectance.
Relative Radiometric Correction: Using multi-temporal images for relative correction to maintain radiometric consistency between images, such as the radiometric normalization method.

Data Fusion.
Data fusion integrates multi-source remote sensing data to enhance overall data performance. Data fusion can be categorized into spatial fusion, temporal fusion, and spectral fusion.

Spatial Fusion\cite{weissRemoteSensingAgricultural2020}.
Spatial fusion combines low spatial resolution data with high spatial resolution data to generate data with both high spatial and high spectral resolution. Here are some methods to do this task.
Principal Component Analysis (PCA): Utilizing principal component analysis, the principal components of high-resolution images are merged with low-resolution multispectral images.
Brovey Transform: Calculating the ratio of each band and merging high-resolution images with low-resolution multispectral images.
Wavelet Transform: Using wavelet transform to merge detailed information from high-resolution images with low-resolution multispectral images.

Temporal Fusion.
Temporal fusion merges remote sensing data acquired at different times to generate data with high temporal resolution, suitable for dynamic monitoring. Here are some methods to do this task\cite{huangAgriculturalRemoteSensing2018}.
Time Series Interpolation Method: Using time series interpolation algorithms (such as linear interpolation and spline interpolation) to merge images from different times, creating continuous time series data.
Data Assimilation Method: Employing data assimilation techniques to merge remote sensing observations with numerical models to improve temporal resolution.

Spectral Fusion.
Spectral fusion merges data with different spectral resolutions to generate data with high spectral resolution, enhancing spectral information richness.
Hyperspectral and Multispectral Fusion: Combining spectral information from hyperspectral images with spatial information from multispectral images, such as the hyperspectral decomposition method.
Multi-band Fusion: Merging multi-band data from different sensors to generate images with rich spectral information. 

Data preprocessing is a crucial step in agricultural remote sensing data analysis, ensuring data accuracy and consistency. Atmospheric correction, geometric correction, and radiometric correction eliminate various errors, enhancing data quality. Data fusion techniques integrate multi-source data to improve spatial, temporal, and spectral resolution, providing a solid data foundation for various agricultural remote sensing applications. By comprehensively utilizing data correction and data fusion techniques, it is possible to achieve fine, dynamic, and comprehensive monitoring of farmland, providing strong technical support for precision agriculture and sustainable agricultural development.