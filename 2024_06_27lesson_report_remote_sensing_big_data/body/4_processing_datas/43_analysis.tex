\subsection{Data Analysis and Model Construction}

In the field of agricultural remote sensing, data analysis and model construction are critical steps for tasks such as crop monitoring, classification, and pest identification. In recent years, the application of machine learning and deep learning technologies in agricultural remote sensing has made significant progress, providing robust technical support for precision agriculture. This section will comprehensively introduce the applications of machine learning and deep learning in agricultural remote sensing, including supervised learning, unsupervised learning, and the use of Convolutional Neural Networks (CNN).

\subsubsection{Applications of Machine Learning in Agricultural Remote Sensing}

Machine learning techniques are widely applied in agricultural remote sensing. They encompass supervised and unsupervised learning methods, each suitable for various application scenarios and data characteristics\cite{weissRemoteSensingAgricultural2020}. Supervised learning utilizes labeled data for training to establish mapping relationships between input data and output labels. Common supervised learning algorithms include Support Vector Machines (SVM), Random Forests (RF), Gradient Boosting Decision Trees (GBDT), among others. These algorithms enable accurate classification of different crop types (e.g., corn, wheat, rice) using multispectral remote sensing data. They can also identify regions affected by pests and diseases on crop leaves, leveraging GBDT models for disease spot detection and pest type identification.

Supervised learning offers high accuracy by leveraging labeled data for training. It provides flexibility in choosing algorithms and features based on specific requirements. 

Nonetheless, unsupervised learning analyzes and clusters data based on internal structures and distribution patterns, bypassing the need for labeled data. K-means clustering, Principal Component Analysis (PCA), and other algorithms facilitate automatic classification of land surfaces based on remote sensing image spectral characteristics. Furthermore, they detect abnormal changes in remote sensing data, such as disease outbreaks or droughts, using PCA algorithms to detect crop growth anomalies.

Unsupervised learning benefits from not requiring extensive labeled data and is applicable in scenarios where labeling is challenging despite large datasets. It allows the automatic discovery of patterns and structures within the data.

\subsubsection{Deep Learning Models}

Deep learning is a prominent branch of machine learning that has achieved remarkable success in image processing and analysis. Convolutional Neural Networks (CNNs) are classic models within deep learning widely used in image classification and object detection tasks.

Convolutional Neural Networks (CNNs) process image data with specialized neural network structures, automatically extracting image features through convolutional layers, pooling layers, and fully connected layers. CNN models utilize multispectral or hyperspectral data to classify various crop types accurately, employing architectures like LeNet, AlexNet, VGG, ResNet, among others.

CNNs achieve high precision in classification tasks by automatically extracting high-order features from images. They support end-to-end learning from raw data to classification results, reducing feature engineering complexity.

Additionally, CNNs' convolutional layers extract local features from images, while fully connected layers integrate these features for complex pattern recognition. By training CNN models with high-resolution images, automatic identification and classification of various pests and diseases on crop leaves can be achieved. CNN architectures such as Inception, DenseNet, MobileNet, among others, enable sensitive detection of detailed image features, ensuring accurate identification of pests and diseases\cite{navalgundRemoteSensingApplications2007}. Moreover, CNNs automate pest and disease identification processes, minimizing manual intervention and enhancing efficiency.

In agricultural remote sensing, data analysis and model construction using machine learning and deep learning technologies are crucial. Supervised and unsupervised learning methods are tailored for tasks with labeled and unlabeled data, respectively, achieving high accuracy in crop classification and anomaly detection. Convolutional Neural Networks (CNNs), as powerful deep learning models, are extensively applied in crop classification and pest identification, offering high accuracy and automation. By integrating these data analysis and model construction techniques, the effectiveness of agricultural remote sensing applications can be significantly enhanced, supporting the advancement of precision agriculture and sustainable farming practices.
