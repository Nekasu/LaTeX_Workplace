
\section{Challenges in Agricultural Remote Sensing}
Agricultural remote sensing has become an indispensable tool in modern precision agriculture, offering numerous benefits for crop monitoring, soil analysis, and environmental management. However, the adoption and implementation of remote sensing technologies in agriculture are not without their challenges. These challenges span various aspects, from data acquisition and processing to technology adoption and farmer acceptance. In this section, we delve into the key challenges faced in agricultural remote sensing. For a quick reference, a summary of these challenges is provided in Table \ref{talble:challenges}.

\subsection{Challenges in Data Acquisition and Processing}

In agricultural remote sensing applications, there are several challenges associated with data acquisition and processing. These challenges primarily include large data volume, uneven data quality, and high acquisition costs. To address these issues, modern information technologies such as cloud computing platforms and distributed computing techniques are essential\cite{begueRemoteSensingCropping2018}.

\subsubsection{Large Data Volume}

Remote sensing data typically exhibit high spatial and temporal resolution and involve multispectral and multi-temporal characteristics, resulting in massive data volumes.


High-resolution remote sensing data provide detailed surface information but also generate huge amounts of data. For example, satellite imagery and UAV (Unmanned Aerial Vehicle)\cite{Unmanned27:online} images generate large datasets daily.


Agricultural remote sensing often requires multispectral and time-series data to monitor crop growth and environmental changes, further increasing data volumes.

Cloud computing platforms offer robust storage and computational capabilities, supporting storage, processing, and analysis of large-scale remote sensing data. For instance, Google Earth Engine provides an efficient cloud platform for online processing and analysis of large-scale remote sensing data, distributed computing technologies distribute large-scale data processing tasks across multiple computing nodes, enhancing data processing efficiency\cite{ozdoganRemoteSensingIrrigated2010}. For example, frameworks like Hadoop and Spark effectively handle and analyze large-scale remote sensing data.

\subsubsection{Uneven Data Quality}

Remote sensing data quality can be affected by factors such as sensor performance, observation conditions, and atmospheric interference, leading to uneven data quality.

Differences in sensor performance can result in varying data quality, such as differences in resolution and signal-to-noise ratio.
Remote sensing data acquisition is significantly influenced by observation conditions such as weather and lighting conditions.
Atmospheric interference (e.g., cloud cover, aerosols) can degrade remote sensing data quality, introducing noise and distortion.

Various data preprocessing techniques can address uneven data quality issues, including radiometric correction, geometric correction, and atmospheric correction to enhance data quality\cite{sahooHyperspectralRemoteSensing2015}. Radiometric correction corrects sensor radiance responses to improve radiometric accuracy, geometric correction corrects geometric distortions in remote sensing images to ensure accurate spatial positioning, atmospheric correction removes atmospheric interference to improve the true reflectance of images, data fusion techniques integrate multiple data sources (e.g., different sensors, observations) to improve data quality and reliability. Common data fusion techniques include image stitching, time-series analysis, and multi-source data fusion.

\subsubsection{High Acquisition Costs}

Acquiring high-resolution remote sensing data involves significant costs, including satellite and UAV remote sensing methods.

The purchase and usage costs of high-resolution satellite imagery are high, especially commercial satellite imagery.

\begin{table*}[h!]
    \centering
    \caption{Challenges in Agricultural Remote Sensing}
    \begin{tabular}{m{6cm}m{9cm}}
    \toprule
    \textbf{Challenge} & \textbf{Description} \\
    \midrule
    Data Acquisition and Processing & 
    \begin{itemize}
        \item \textbf{Large Data Volumes}: Managing and processing large volumes of remote sensing data.
        \item \textbf{Uneven Data Quality}: Variability in data quality due to sensor performance, observation conditions, and atmospheric interference.
        \item \textbf{High Acquisition Costs}: Significant costs associated with acquiring high-resolution data from satellites, UAVs, and ground sensors.
    \end{itemize} \\
    \midrule
    Technology Adoption and Farmer Acceptance & 
    \begin{itemize}
        \item \textbf{Limited Awareness and Knowledge}: Many farmers are unaware of remote sensing technologies and their benefits.
        \item \textbf{Perceived Complexity}: The complexity of remote sensing processes can intimidate farmers.
        \item \textbf{Cost Concerns}: High initial costs for adopting remote sensing technologies.
    \end{itemize} \\
    \bottomrule
    \end{tabular}
    \label{table:challenges}
\end{table*}

Although UAV remote sensing offers high flexibility, it involves costs related to equipment purchase, maintenance, and operation.

Acquiring data from ground sensors also incurs equipment and maintenance costs, particularly in large-scale monitoring.

Leveraging free and open-source remote sensing data such as Landsat and Sentinel satellite data reduces data acquisition costs. These datasets provide rich, high-quality remote sensing images, lowering acquisition costs, establishing remote sensing data sharing platforms facilitates data resource sharing and utilization. Through sharing platforms, different institutions and research teams can collaborate, reducing redundant acquisition costs, employing automation technology for data acquisition and processing, such as autonomous UAV flights and automated data download and processing, enhances data acquisition and processing efficiency, reducing labor costs.

Data acquisition and processing are critical aspects of agricultural remote sensing applications, facing challenges such as large data volumes, uneven data quality, and high acquisition costs. By utilizing solutions like cloud computing platforms, distributed computing techniques, data preprocessing and fusion techniques, free data sources, sharing platforms, and automation technology, these challenges can be effectively addressed. This will enhance the efficiency and quality of data acquisition and processing in agricultural remote sensing, providing reliable data support for precision agriculture, and improving agricultural productivity and sustainability.

