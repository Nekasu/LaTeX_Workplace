\documentclass{standalone}
% \documentclass{paper}
\usepackage{D:/DevData/LaTeX_Workplace/2_macro_definition/network_tikz_package}
\RequirePackage[CJKnumber=true]{xeCJK}
\begin{document}

\begin{tikzpicture}
    % \draw[help lines] (-3,-10) grid (20,10);
    % Use the new command to draw a single rounded rectangle
    % \rSquare{0.6}{NekasuBlue}{20}{5pt}{0}{0}{rs1}

    %表示图像中两个元素之间的水平偏移
    \def\offX{2}
    %表示图像中两个元素之间的垂直偏移
    \def\offY{3}
    %创建一个方形, 本阶段输入1:Rcei
    \coordinate (Rcei) at ($(0,0)+(0, 4*\offY/4)$);
    \def\widthRcei{1.2};
    \coordinate (Rcei_left_bottom) at ($(Rcei)+(-\widthRcei/2,-\widthRcei/2)$);
    \coordinate (Rcei_right_bottom) at ($(Rcei)+(\widthRcei/2,\widthRcei/2)$);
    \coordinate (Rcei_left) at ($(Rcei)+(-\widthRcei/2-0.1,0)$);
    \coordinate (Rcei_right) at ($(Rcei)+(\widthRcei/2+0.1,0)$);
    \coordinate (Rcei_upper) at ($(Rcei)+(0,\widthRcei/2+0.1)$);
    \coordinate (Rcei_bottom) at ($(Rcei)+(0,-\widthRcei/2-0.1)$);
    \draw[fill=GanYu_ricewhite!30] (Rcei_left_bottom) rectangle (Rcei_right_bottom);
    \node[align=center] at ($(Rcei)+(0,-\widthRcei/2-0.5)$) (Rcei_text) {内容兴趣对象\\$R_{ce\_i}$};

    %创建一个方形, 本阶段输入2:Sm
    \coordinate (Sm) at ($(0,0)+(0, -4*\offY/4)$);
    \def\widthSm{1.2};
    \coordinate (Sm_left_bottom) at ($(Sm)+(-\widthSm/2,-\widthSm/2)$);
    \coordinate (Sm_right_bottom) at ($(Sm)+(\widthSm/2,\widthSm/2)$);
    \coordinate (Sm_left) at ($(Sm)+(-\widthSm/2-0.1,0)$);
    \coordinate (Sm_right) at ($(Sm)+(\widthSm/2+0.1,0)$);
    \coordinate (Sm_upper) at ($(Sm)+(0,\widthSm/2+0.1)$);
    \coordinate (Sm_bottom) at ($(Sm)+(0,-\widthSm/2-0.1)$);
    \draw[fill=GanYu_ricewhite!30] (Sm_left_bottom) rectangle (Sm_right_bottom);
    \node[align=center] at ($(Sm)+(0,-\widthSm/2-0.5)$) (Sm_text) {风格主体对象\\$S_m$};

    %创建一个神经网络, 用于表示八度卷积OctConv1, 并定义一个坐标表示这个模块
    \coordinate (OctConv1) at ($(Rcei)+(\offX,0)$);
    \networkLayer{1.75}{0.15}{0.0}{0.0}{0.0}{color=GanYu_lightblue!40}{}{start}{(3,4*\offY/4,0)};
    \networkLayer{1}{0.15}{0.1}{0.0}{0.0}{color=GanYu_lightblue!50}{}{}{(3,4*\offY/4,0)};
    \networkLayer{1.25}{0.15}{0.1}{0.0}{0.0}{color=GanYu_lightblue!70}{}{mid_right}{(3,4*\offY/4,0)};
    \networkLayer{1}{0.15}{0.2}{0.0}{0.0}{color=GanYu_lightblue!40}{}{last}{(3,4*\offY/4,0)};
    \node[align=center] at ($(mid_right_bottom)+(-0.08,-1)$) (OctConv1_bottom) {八度卷积\\$OctConv$};

    %创建一个神经网络, 用于表示八度卷积OctConv2, 并定义一个坐标表示这个模块
    \coordinate (OctConv2) at ($(Sm)+(\offX,0)$);
    \networkLayer{1.75}{0.15}{0.0}{0.0}{0.0}{color=GanYu_lightblue!40}{}{start}{(3,-4*\offY/4,0)};
    \networkLayer{1}{0.15}{0.1}{0.0}{0.0}{color=GanYu_lightblue!50}{}{}{(3,-4*\offY/4,0)};
    \networkLayer{1.25}{0.15}{0.1}{0.0}{0.0}{color=GanYu_lightblue!70}{}{mid_right}{(3,-4*\offY/4,0)};
    \networkLayer{1}{0.15}{0.2}{0.0}{0.0}{color=GanYu_lightblue!40}{}{last}{(3,-4*\offY/4,0)};
    \node[align=center] at ($(mid_right_bottom)+(-0.08,-1)$) (OctConv2_bottom) {八度卷积\\$OctConv$};

    %创建一个矩形, 用于包裹RceiH与RceiL
    \draw[fill=6ecd32!10, rounded corners=5pt] (5.5,0)rectangle(8.5,5);
    \coordinate (verticleRec1_left) at (5.3,4*\offY/4);
    \coordinate (verticleRec1_right) at (8.7,4*\offY/4);
    %创建一个方形, RceiH
    \coordinate (RceiH) at ($(Rcei)+(3.5*\offX, \offY/2.5)$);
    \def\widthRceiH{1.2};
    \coordinate (RceiH_left_bottom) at ($(RceiH)+(-\widthRceiH/2,-\widthRceiH/2)$);
    \coordinate (RceiH_right_bottom) at ($(RceiH)+(\widthRceiH/2,\widthRceiH/2)$);
    \coordinate (RceiH_left) at ($(RceiH)+(-\widthRceiH/2-0.1,0)$);
    \coordinate (RceiH_right) at ($(RceiH)+(\widthRceiH/2+0.1,0)$);
    \coordinate (RceiH_upper) at ($(RceiH)+(0,\widthRceiH/2+0.1)$);
    \coordinate (RceiH_bottom) at ($(RceiH)+(0,-\widthRceiH/2-0.1)$);
    \draw[fill=GanYu_ricewhite!30] (RceiH_left_bottom) rectangle (RceiH_right_bottom);
    \node[align=center] at ($(RceiH)+(0,-\widthRceiH/2-0.7)$) (RceiH_text) {内容兴趣对象\\高频分量$R_{ce\_i}^H$};

    %创建一个方形, RceiL
    \coordinate (RceiL) at ($(Rcei)+(3.5*\offX, -\offY/2.5)$);
    \def\widthRceiL{1.2};
    \coordinate (RceiL_left_bottom) at ($(RceiL)+(-\widthRceiL/2,-\widthRceiL/2)$);
    \coordinate (RceiL_right_bottom) at ($(RceiL)+(\widthRceiL/2,\widthRceiL/2)$);
    \coordinate (RceiL_left) at ($(RceiL)+(-\widthRceiL/2-0.1,0)$);
    \coordinate (RceiL_right) at ($(RceiL)+(\widthRceiL/2+0.1,0)$);
    \coordinate (RceiL_upper) at ($(RceiL)+(0,\widthRceiL/2+0.1)$);
    \coordinate (RceiL_bottom) at ($(RceiL)+(0,-\widthRceiL/2-0.1)$);
    \draw[fill=GanYu_ricewhite!30] (RceiL_left_bottom) rectangle (RceiL_right_bottom);
    \node[align=center] at ($(RceiL)+(0,-\widthRceiL/2-0.7)$) (RceiL_text) {内容兴趣对象\\低频分量$R_{ce\_i}^L$};

    %创建一个矩形, 用于包裹SmH与SmL
    \draw[fill=6ecd32!10, rounded corners=5pt] (5.5,-5.9)rectangle(8.5,-0.9);
    \coordinate (verticleRec2_left) at (5.3,-4*\offY/4);
    \coordinate (verticleRec2_right) at (8.7,-4*\offY/4);
    %创建一个方形, SmH
    \coordinate (SmH) at ($(Sm)+(3.5*\offX, \offY/2.5)$);
    \def\widthSmH{1.2};
    \coordinate (SmH_left_bottom) at ($(SmH)+(-\widthSmH/2,-\widthSmH/2)$);
    \coordinate (SmH_right_bottom) at ($(SmH)+(\widthSmH/2,\widthSmH/2)$);
    \coordinate (SmH_left) at ($(SmH)+(-\widthSmH/2-0.1,0)$);
    \coordinate (SmH_right) at ($(SmH)+(\widthSmH/2+0.1,0)$);
    \coordinate (SmH_upper) at ($(SmH)+(0,\widthSmH/2+0.1)$);
    \coordinate (SmH_bottom) at ($(SmH)+(0,-\widthSmH/2-0.1)$);
    \draw[fill=GanYu_ricewhite!30] (SmH_left_bottom) rectangle (SmH_right_bottom);
    \node[align=center] at ($(SmH)+(0,-\widthSmH/2-0.7)$) (SmH_text) {风格主体对象\\高频分量$S_m^H$};

    %创建一个方形, SmL
    \coordinate (SmL) at ($(Sm)+(3.5*\offX, -\offY/2.5)$);
    \def\widthSmL{1.2};
    \coordinate (SmL_left_bottom) at ($(SmL)+(-\widthSmL/2,-\widthSmL/2)$);
    \coordinate (SmL_right_bottom) at ($(SmL)+(\widthSmL/2,\widthSmL/2)$);
    \coordinate (SmL_left) at ($(SmL)+(-\widthSmL/2-0.1,0)$);
    \coordinate (SmL_right) at ($(SmL)+(\widthSmL/2+0.1,0)$);
    \coordinate (SmL_upper) at ($(SmL)+(0,\widthSmL/2+0.1)$);
    \coordinate (SmL_bottom) at ($(SmL)+(0,-\widthSmL/2-0.1)$);
    \draw[fill=GanYu_ricewhite!30] (SmL_left_bottom) rectangle (SmL_right_bottom);
    \node[align=center] at ($(SmL)+(0,-\widthSmL/2-0.7)$) (SmL_text) {风格主体对象\\低频分量$S_m^L$};

    %创建一个神经网络, 用于表示Es, 并定义一个坐标表示这个模块
    \coordinate (Es) at ($(Sm)+(4.75*\offX,0)$);
    \networkLayer{0.9}{0.15}{0.0}{0.0}{0.0}{color=Nahida_darkgreen!40}{}{start}{(4.75*\offX,-4*\offY/4,0)};
    \networkLayer{1.46}{0.15}{0.1}{0.0}{0.0}{color=Nahida_darkgreen!50}{}{}{(4.75*\offX,-4*\offY/4,0)};
    \networkLayer{0.67}{0.15}{0.1}{0.0}{0.0}{color=Nahida_darkgreen!70}{}{mid_right}{(4.75*\offX,-4*\offY/4,0)};
    \networkLayer{1.41}{0.15}{0.2}{0.0}{0.0}{color=Nahida_darkgreen!40}{}{last}{(4.75*\offX,-4*\offY/4,0)};
    \node[align=center] at ($(mid_right_bottom)+(-0.08,-1.2)$) (Es_bottom) {风格特征提取器\\$E_s$};

    %创建一个矩形, 用于包裹knH与bnH
    \draw[fill=Nahida_lightgreen!10, rounded corners=5pt] (12,-3.1)rectangle(16,-0.7);
    \coordinate (horizonRec1_left) at (11.8,-2);
    \coordinate (horizonRec1_right) at (16.2,-2);

    %创建一个方形, 用于表示knH, 是一个矩阵
    \coordinate (knH) at ($(Sm)+(6.5*\offX,1.5*\offY/3)$);
    \def\widthknH{1.2};
    \coordinate (knH_left_bottom) at ($(knH)+(-\widthknH/2,-\widthknH/2)$);
    \coordinate (knH_right_bottom) at ($(knH)+(\widthknH/2,\widthknH/2)$);
    \coordinate (knH_left) at ($(knH)+(-\widthknH/2-0.1,0)$);
    \coordinate (knH_right) at ($(knH)+(\widthknH/2+0.1,0)$);
    \coordinate (knH_upper) at ($(knH)+(0,\widthknH/2+0.1)$);
    \coordinate (knH_bottom) at ($(knH)+(0,-\widthknH/2-0.1)$);
    \draw (knH_left_bottom) rectangle (knH_right_bottom);
    \node[align=center] at ($(knH)+(0,-\widthknH/2-0.5)$) (knH_text) {高频卷积核\\$k_{n,H}$};
    %为knH的方格涂上颜色
        %(0,0)
    \fill[NekasuBlue!76] ($(knH_left_bottom)+(\widthknH/3*0, \widthknH/3*0)$) rectangle ($(knH_left_bottom)+(\widthknH/3*0+\widthknH/3, \widthknH/3*0+\widthknH/3)$);
        %(1,0)  
    \fill[GanYu_lightblue!90] ($(knH_left_bottom)+(\widthknH/3*0, \widthknH/3*1)$) rectangle ($(knH_left_bottom)+(\widthknH/3*0+\widthknH/3, \widthknH/3*1+\widthknH/3)$);
        %(2,0)
    \fill[NekasuBlue!63] ($(knH_left_bottom)+(\widthknH/3*0, \widthknH/3*2)$) rectangle ($(knH_left_bottom)+(\widthknH/3*0+\widthknH/3, \widthknH/3*2+\widthknH/3)$);
        %(1,0)
    \fill[GanYu_lightblue!81] ($(knH_left_bottom)+(\widthknH/3*1, \widthknH/3*0)$) rectangle ($(knH_left_bottom)+(\widthknH/3*1+\widthknH/3, \widthknH/3*0+\widthknH/3)$);
        %(1,1)
    \fill[GanYu_lightblue!94] ($(knH_left_bottom)+(\widthknH/3*1, \widthknH/3*1)$) rectangle ($(knH_left_bottom)+(\widthknH/3*1+\widthknH/3, \widthknH/3*1+\widthknH/3)$);
        %(1,2)
    \fill[GanYu_lightblue!62] ($(knH_left_bottom)+(\widthknH/3*1, \widthknH/3*2)$) rectangle ($(knH_left_bottom)+(\widthknH/3*1+\widthknH/3, \widthknH/3*2+\widthknH/3)$);
        %(2,0)
    \fill[GanYu_lightblue!86] ($(knH_left_bottom)+(\widthknH/3*2, \widthknH/3*0)$) rectangle ($(knH_left_bottom)+(\widthknH/3*2+\widthknH/3, \widthknH/3*0+\widthknH/3)$);
        %(2,1)
    \fill[GanYu_lightblue!75] ($(knH_left_bottom)+(\widthknH/3*2, \widthknH/3*1)$) rectangle ($(knH_left_bottom)+(\widthknH/3*2+\widthknH/3, \widthknH/3*1+\widthknH/3)$);
        %(2,2)
    \fill[GanYu_lightblue!90] ($(knH_left_bottom)+(\widthknH/3*2, \widthknH/3*2)$) rectangle ($(knH_left_bottom)+(\widthknH/3*2+\widthknH/3, \widthknH/3*2+\widthknH/3)$);
    % 为knH划上方格, 表示一个矩阵
    \foreach \y in {0, \widthknH/3, 2*\widthknH/3}{
        \draw[GanYu_lightgray,thin] ($(knH_left_bottom)+(0,\y)$) -- ($(knH_left_bottom)+(\widthknH,\y)$);
    }
    \foreach \x in {0, \widthknH/3, 2*\widthknH/3}{
        \draw[GanYu_lightgray, thin] ($(knH_left_bottom)+(\x,0)$) -- ($(knH_left_bottom)+(\x,\widthknH)$);
    }
    
    %创建一个矩形, 表示表bnH
    \coordinate (bnH) at ($(knH)+(2,0)$);
    \def\widthbnH{0.4};
    \def\lengthbnH{1.2};
    \coordinate (bnH_left_bottom) at ($(bnH)+(-\widthbnH/2,-\lengthbnH/2)$);
    \coordinate (bnH_right_bottom) at ($(bnH)+(\widthbnH/2,\lengthbnH/2)$);
    \coordinate (bnH_left) at ($(bnH)+(-\lengthbnH/2-0.1,0)$);
    \coordinate (bnH_right) at ($(bnH)+(\lengthbnH/2+0.1,0)$);
    \coordinate (bnH_upper) at ($(bnH)+(0,\lengthbnH/2+0.1)$);
    \coordinate (bnH_bottom) at ($(bnH)+(0,-\lengthbnH/2-0.1)$);
    \draw (bnH_left_bottom) rectangle (bnH_right_bottom);
    \node[align=center] at ($(bnH)+(0,-\lengthbnH/2-0.5)$) (bnH_text) {高频偏置项\\$b_{n,H}$};
    %为bnH上色
    \fill[NekasuBlue!76] ($(bnH_left_bottom)+(\widthbnH*0, \lengthbnH/3*0)$) rectangle ($(bnH_left_bottom)+(\widthbnH*0+\widthbnH, \lengthbnH/3*0+\lengthbnH/3)$);
    \fill[GanYu_lightblue!94] ($(bnH_left_bottom)+(\widthbnH*0, \lengthbnH/3*1)$) rectangle ($(bnH_left_bottom)+(\widthbnH*0+\widthbnH, \lengthbnH/3*1+\lengthbnH/3)$);
    \fill[GanYu_lightgray!90] ($(bnH_left_bottom)+(\widthbnH*0, \lengthbnH/3*2)$) rectangle ($(bnH_left_bottom)+(\widthbnH*0+\widthbnH, \lengthbnH/3*2+\lengthbnH/3)$);

    %创建一个矩形, 用于包裹knL与bnL
    \draw[fill=Nahida_lightgreen!10, rounded corners=5pt] (12,-5.6)rectangle(16,-3.2);
    \coordinate (horizonRec2_left) at (11.8,-4);
    \coordinate (horizonRec2_right) at (16.2,-4);
    %创建一个方形, 用于表示knL, 是一个矩阵
    \coordinate (knL) at ($(Sm)+(6.5*\offX,-\offY/3)$);
    \def\widthknL{1.2};
    \coordinate (knL_left_bottom) at ($(knL)+(-\widthknL/2,-\widthknL/2)$);
    \coordinate (knL_right_bottom) at ($(knL)+(\widthknL/2,\widthknL/2)$);
    \coordinate (knL_left) at ($(knL)+(-\widthknL/2-0.1,0)$);
    \coordinate (knL_right) at ($(knL)+(\widthknL/2+0.1,0)$);
    \coordinate (knL_upper) at ($(knL)+(0,\widthknL/2+0.1)$);
    \coordinate (knL_bottom) at ($(knL)+(0,-\widthknL/2-0.1)$);
    \draw (knL_left_bottom) rectangle (knL_right_bottom);
    \node[align=center] at ($(knL)+(0,-\widthknL/2-0.5)$) (knL_text) {高频卷积核\\$k_{n,L}$};
    %为knL的方格涂上颜色
        %(0,0)
    \fill[NekasuBlue!76] ($(knL_left_bottom)+(\widthknL/3*0, \widthknL/3*0)$) rectangle ($(knL_left_bottom)+(\widthknL/3*0+\widthknL/3, \widthknL/3*0+\widthknL/3)$);
        %(1,0)  
    \fill[GanYu_lightblue!90] ($(knL_left_bottom)+(\widthknL/3*0, \widthknL/3*1)$) rectangle ($(knL_left_bottom)+(\widthknL/3*0+\widthknL/3, \widthknL/3*1+\widthknL/3)$);
        %(2,0)
    \fill[NekasuBlue!63] ($(knL_left_bottom)+(\widthknL/3*0, \widthknL/3*2)$) rectangle ($(knL_left_bottom)+(\widthknL/3*0+\widthknL/3, \widthknL/3*2+\widthknL/3)$);
        %(1,0)
    \fill[GanYu_lightblue!81] ($(knL_left_bottom)+(\widthknL/3*1, \widthknL/3*0)$) rectangle ($(knL_left_bottom)+(\widthknL/3*1+\widthknL/3, \widthknL/3*0+\widthknL/3)$);
        %(1,1)
    \fill[GanYu_lightblue!94] ($(knL_left_bottom)+(\widthknL/3*1, \widthknL/3*1)$) rectangle ($(knL_left_bottom)+(\widthknL/3*1+\widthknL/3, \widthknL/3*1+\widthknL/3)$);
        %(1,2)
    \fill[GanYu_lightblue!62] ($(knL_left_bottom)+(\widthknL/3*1, \widthknL/3*2)$) rectangle ($(knL_left_bottom)+(\widthknL/3*1+\widthknL/3, \widthknL/3*2+\widthknL/3)$);
        %(2,0)
    \fill[GanYu_lightblue!86] ($(knL_left_bottom)+(\widthknL/3*2, \widthknL/3*0)$) rectangle ($(knL_left_bottom)+(\widthknL/3*2+\widthknL/3, \widthknL/3*0+\widthknL/3)$);
        %(2,1)
    \fill[GanYu_lightblue!75] ($(knL_left_bottom)+(\widthknL/3*2, \widthknL/3*1)$) rectangle ($(knL_left_bottom)+(\widthknL/3*2+\widthknL/3, \widthknL/3*1+\widthknL/3)$);
        %(2,2)
    \fill[GanYu_lightblue!90] ($(knL_left_bottom)+(\widthknL/3*2, \widthknL/3*2)$) rectangle ($(knL_left_bottom)+(\widthknL/3*2+\widthknL/3, \widthknL/3*2+\widthknL/3)$);
    % 为knL划上方格, 表示一个矩阵
    \foreach \y in {0, \widthknL/3, 2*\widthknL/3}{
        \draw[GanYu_lightgray,thin] ($(knL_left_bottom)+(0,\y)$) -- ($(knL_left_bottom)+(\widthknL,\y)$);
    }
    \foreach \x in {0, \widthknL/3, 2*\widthknL/3}{
        \draw[GanYu_lightgray, thin] ($(knL_left_bottom)+(\x,0)$) -- ($(knL_left_bottom)+(\x,\widthknL)$);
    }

    %创建一个矩形, 表示表bnL
    \coordinate (bnL) at ($(knL)+(2,0)$);
    \def\widthbnL{0.4};
    \def\lengthbnL{1.2};
    \coordinate (bnL_left_bottom) at ($(bnL)+(-\widthbnL/2,-\lengthbnL/2)$);
    \coordinate (bnL_right_bottom) at ($(bnL)+(\widthbnL/2,\lengthbnL/2)$);
    \coordinate (bnL_left) at ($(bnL)+(-\lengthbnL/2-0.1,0)$);
    \coordinate (bnL_right) at ($(bnL)+(\lengthbnL/2+0.1,0)$);
    \coordinate (bnL_upper) at ($(bnL)+(0,\lengthbnL/2+0.1)$);
    \coordinate (bnL_bottom) at ($(bnL)+(0,-\lengthbnL/2-0.1)$);
    \draw (bnL_left_bottom) rectangle (bnL_right_bottom);
    \node[align=center] at ($(bnL)+(0,-\lengthbnL/2-0.5)$) (bnL_text) {低频偏置项\\$b_{n,L}$};
    %为bnL上色
    \fill[NekasuBlue!76] ($(bnL_left_bottom)+(\widthbnL*0, \lengthbnL/3*0)$) rectangle ($(bnL_left_bottom)+(\widthbnL*0+\widthbnL, \lengthbnL/3*0+\lengthbnL/3)$);
    \fill[GanYu_lightgray!94] ($(bnL_left_bottom)+(\widthbnL*0, \lengthbnL/3*1)$) rectangle ($(bnL_left_bottom)+(\widthbnL*0+\widthbnL, \lengthbnL/3*1+\lengthbnL/3)$);
    \fill[GanYu_lightblue!90] ($(bnL_left_bottom)+(\widthbnL*0, \lengthbnL/3*2)$) rectangle ($(bnL_left_bottom)+(\widthbnL*0+\widthbnL, \lengthbnL/3*2+\lengthbnL/3)$);

    %创建一个神经网络, 用于表示G, 并定义一个坐标表示这个模块
    \coordinate (G) at ($(Sm)+(9*\offX,0)$);
    \networkLayer{1.36}{0.15}{0.0}{0.0}{0.0}{color=Nahida_darkgreen!40}{}{start}{(9*\offX,-4*\offY/4,0)};
    \networkLayer{1.01}{0.15}{0.1}{0.0}{0.0}{color=Nahida_darkgreen!50}{}{}{(9*\offX,-4*\offY/4,0)};
    \networkLayer{1.18}{0.15}{0.1}{0.0}{0.0}{color=Nahida_darkgreen!70}{}{mid_right}{(9*\offX,-4*\offY/4,0)};
    \networkLayer{1.46}{0.15}{0.2}{0.0}{0.0}{color=Nahida_darkgreen!40}{}{last}{(9*\offX,-4*\offY/4,0)};
    \node[align=center] at ($(mid_right_bottom)+(-0.08,-1)$) (G_bottom) {风格迁移生成器\\$G$};


    %创建一个方形, 本阶段输出:Om
    \coordinate (Om) at ($(Sm)+(10.5*\offX, 0)$);
    \def\widthOm{1.2};
    \coordinate (Om_left_bottom) at ($(Om)+(-\widthOm/2,-\widthOm/2)$);
    \coordinate (Om_right_bottom) at ($(Om)+(\widthOm/2,\widthOm/2)$);
    \coordinate (Om_left) at ($(Om)+(-\widthOm/2-0.1,0)$);
    \coordinate (Om_right) at ($(Om)+(\widthOm/2+0.1,0)$);
    \coordinate (Om_upper) at ($(Om)+(0,\widthOm/2+0.1)$);
    \coordinate (Om_bottom) at ($(Om)+(0,-\widthOm/2-0.1)$);
    \draw[fill=GanYu_ricewhite!30!GanYu_lightblue!90] (Om_left_bottom) rectangle (Om_right_bottom);
    \node[align=center] at ($(Om)+(0,-\widthOm/2-0.5)$) (Om_text) {风格化主体对象\\$O_m$};

    % 绘制箭头
    % CRcei上的出度箭头1:Rcei->Octconv1
    \draw[->, thick] (Rcei_right) -- (2.5,4*\offY/4);

    % Sm上的出度箭头1:Sm->Octconv2
    \draw[->, thick] (Sm_right) -- (2.3,-4*\offY/4);

    %OctConv1上的出度箭头1:OctCOnv1=>verticleRec1
    \draw[->, thick] (4.3,4*\offY/4) -- (verticleRec1_left);

    %OctConv2上的出度箭头1:OctCOnv2=>verticleRec2
    \draw[->, thick] (4.3,-4*\offY/4) -- (verticleRec2_left);

    %verticleRec1上的出度箭头1:verticleRec1->G
    \draw[->, thick] (verticleRec1_right) -| (15.5,-1);

    %verticleRec2上的出度箭头1:verticleRec2->Es
    \draw[->, thick] (verticleRec2_right) -- ($(verticleRec2_right)+(0.4,0)$);

    %Es 上的出度箭头1: Es->horizonRec1
    \draw[->, thick] ($(verticleRec2_right)+(2.4,0)$) -- ($(verticleRec2_right)+(2.6,0)$) |- (horizonRec1_left);
    %Es 上的出度箭头2: Es->horizonRec2
    \draw[->, thick] ($(verticleRec2_right)+(2.4,0)$) -- ($(verticleRec2_right)+(2.6,0)$) |- (horizonRec2_left);

    %horizonRec1 上的出度箭头1: horizonRec1 -> G
    \draw[->, thick] (horizonRec1_right) -- ($(horizonRec1_right)+(0.3,0)$) |- (17.5,-4*\offY/4);

    %horizonRec2 上的出度箭头1: horizonRec2 -> G
    \draw[->, thick] (horizonRec2_right) -- ($(horizonRec2_right)+(0.3,0)$) |- (17.5,-4*\offY/4);

    % G 上的出度箭头1:G->Om
    \draw[->, thick] (19.5,-4*\offY/4) -- (Om_left);

\end{tikzpicture}

\end{document}
