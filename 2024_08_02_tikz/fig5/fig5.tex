\documentclass{standalone}
% \documentclass{paper}
\usepackage{D:/DevData/LaTeX_Workplace/2_macro_definition/network_tikz_package}
\RequirePackage[CJKnumber=true]{xeCJK}
\begin{document}

\begin{tikzpicture}
    % \draw[help lines] (-3,-10) grid (20,10);
        % Use the new command to draw a single rounded rectangle
    % \rSquare{0.6}{NekasuBlue}{20}{5pt}{0}{0}{rs1}

    %表示图像中两个元素之间的水平偏移
    \def\offX{3}
    %表示图像中两个元素之间的垂直偏移
    \def\offY{2.5}

    %创建一个方形, 本阶段输入1: Om
    \coordinate (Om) at ($(0,0)+(0, 0)$);
    \def\widthOm{1.2};
    \coordinate (Om_left_bottom) at ($(Om)+(-\widthOm/2,-\widthOm/2)$);
    \coordinate (Om_right_bottom) at ($(Om)+(\widthOm/2,\widthOm/2)$);
    \coordinate (Om_left) at ($(Om)+(-\widthOm/2-0.1,0)$);
    \coordinate (Om_right) at ($(Om)+(\widthOm/2+0.1,0)$);
    \coordinate (Om_upper) at ($(Om)+(0,\widthOm/2+0.1)$);
    \coordinate (Om_bottom) at ($(Om)+(0,-\widthOm/2-0.1)$);
    \draw[fill=GanYu_ricewhite!30!GanYu_lightblue!90] (Om_left_bottom) rectangle (Om_right_bottom);
    \node[align=center] at ($(Om)+(0,-\widthOm/2-0.5)$) (Om_text) {风格化主体对象\\$O_m$};

    %创建一个方形, 本阶段输入2: Oo
    \coordinate (Oo) at ($(Om)+(0, -\offY)$);
    \def\widthOo{1.2};
    \coordinate (Oo_left_bottom) at ($(Oo)+(-\widthOo/2,-\widthOo/2)$);
    \coordinate (Oo_right_bottom) at ($(Oo)+(\widthOo/2,\widthOo/2)$);
    \coordinate (Oo_left) at ($(Oo)+(-\widthOo/2-0.1,0)$);
    \coordinate (Oo_right) at ($(Oo)+(\widthOo/2+0.1,0)$);
    \coordinate (Oo_upper) at ($(Oo)+(0,\widthOo/2+0.1)$);
    \coordinate (Oo_bottom) at ($(Oo)+(0,-\widthOo/2-0.1)$);
    \draw[fill=Nahida_darkgreen!50!green!30] (Oo_left_bottom) rectangle (Oo_right_bottom);
    \node[align=center] at ($(Oo)+(0,-\widthOo/2-0.5)$) (Oo_text) {风格化背景对象\\$O_{o}$};

    %创建一个odot
    \coordinate (odot1) at ($(Om)+(\offX, 0)$);
    \node at (odot1) (odot1) {\large $\odot$};

    %创建一个方形, 用于表示Mi, 是一个矩阵
    \coordinate (Mi) at ($(odot1)+(0,1+2*\offY/3)$);
    \def\widthMi{1.2};
    \coordinate (Mi_left_bottom) at ($(Mi)+(-\widthMi/2,-\widthMi/2)$);
    \coordinate (Mi_right_bottom) at ($(Mi)+(\widthMi/2,\widthMi/2)$);
    \coordinate (Mi_left) at ($(Mi)+(-\widthMi/2-0.1,0)$);
    \coordinate (Mi_right) at ($(Mi)+(\widthMi/2+0.1,0)$);
    \coordinate (Mi_upper) at ($(Mi)+(0,\widthMi/2+0.1)$);
    \coordinate (Mi_bottom) at ($(Mi)+(0,-\widthMi/2-0.1)$);
    \draw (Mi_left_bottom) rectangle (Mi_right_bottom);
    \node[align=center] at ($(Mi)+(0,-\widthMi/2-0.5)$) (Mi_text) {兴趣掩膜及其$\alpha$通道\\$[M_{i},M_{i}^{\alpha}]$};
    %为Mi的方格涂上颜色
    %(0,0)
    \fill[NekasuBlue!76] ($(Mi_left_bottom)+(\widthMi/3*0, \widthMi/3*0)$) rectangle ($(Mi_left_bottom)+(\widthMi/3*0+\widthMi/3, \widthMi/3*0+\widthMi/3)$);
        %(1,0)  
    \fill[GanYu_midblue!90] ($(Mi_left_bottom)+(\widthMi/3*0, \widthMi/3*1)$) rectangle ($(Mi_left_bottom)+(\widthMi/3*0+\widthMi/3, \widthMi/3*1+\widthMi/3)$);
        %(2,0)
    \fill[NekasuBlue!63] ($(Mi_left_bottom)+(\widthMi/3*0, \widthMi/3*2)$) rectangle ($(Mi_left_bottom)+(\widthMi/3*0+\widthMi/3, \widthMi/3*2+\widthMi/3)$);
        %(1,0)
    \fill[GanYu_lightblue!81] ($(Mi_left_bottom)+(\widthMi/3*1, \widthMi/3*0)$) rectangle ($(Mi_left_bottom)+(\widthMi/3*1+\widthMi/3, \widthMi/3*0+\widthMi/3)$);
        %(1,1)
    \fill[GanYu_lightblue!94] ($(Mi_left_bottom)+(\widthMi/3*1, \widthMi/3*1)$) rectangle ($(Mi_left_bottom)+(\widthMi/3*1+\widthMi/3, \widthMi/3*1+\widthMi/3)$);
        %(1,2)
    \fill[GanYu_lightblue!62] ($(Mi_left_bottom)+(\widthMi/3*1, \widthMi/3*2)$) rectangle ($(Mi_left_bottom)+(\widthMi/3*1+\widthMi/3, \widthMi/3*2+\widthMi/3)$);
        %(2,0)
    \fill[GanYu_midblue!86] ($(Mi_left_bottom)+(\widthMi/3*2, \widthMi/3*0)$) rectangle ($(Mi_left_bottom)+(\widthMi/3*2+\widthMi/3, \widthMi/3*0+\widthMi/3)$);
        %(2,1)
    \fill[GanYu_midblue!75] ($(Mi_left_bottom)+(\widthMi/3*2, \widthMi/3*1)$) rectangle ($(Mi_left_bottom)+(\widthMi/3*2+\widthMi/3, \widthMi/3*1+\widthMi/3)$);
        %(2,2)
    \fill[GanYu_midblue!90] ($(Mi_left_bottom)+(\widthMi/3*2, \widthMi/3*2)$) rectangle ($(Mi_left_bottom)+(\widthMi/3*2+\widthMi/3, \widthMi/3*2+\widthMi/3)$);
    % 为Mi划上方格, 表示一个矩阵
    \foreach \y in {0, \widthMi/3, 2*\widthMi/3}{
        \draw[GanYu_lightgray,thin] ($(Mi_left_bottom)+(0,\y)$) -- ($(Mi_left_bottom)+(\widthMi,\y)$);
    }
    \foreach \x in {0, \widthMi/3, 2*\widthMi/3}{
        \draw[GanYu_lightgray, thin] ($(Mi_left_bottom)+(\x,0)$) -- ($(Mi_left_bottom)+(\x,\widthMi)$);
    }

    %创建一个odot
    \coordinate (odot2) at ($(Oo)+(2*\offX, 0)$);
    \node at (odot2) (odot2) {\large $\odot$};

    %创建一个方形, 用于表示Mo, 是一个矩阵
    \coordinate (Mo) at ($(odot2)+(0,-2*\offY/3-0.6)$);
    \def\widthMo{1.2};
    \coordinate (Mo_left_bottom) at ($(Mo)+(-\widthMo/2,-\widthMo/2)$);
    \coordinate (Mo_right_bottom) at ($(Mo)+(\widthMo/2,\widthMo/2)$);
    \coordinate (Mo_left) at ($(Mo)+(-\widthMo/2-0.1,0)$);
    \coordinate (Mo_right) at ($(Mo)+(\widthMo/2+0.1,0)$);
    \coordinate (Mo_upper) at ($(Mo)+(0,\widthMo/2+0.1)$);
    \coordinate (Mo_bottom) at ($(Mo)+(0,-\widthMo/2-0.1)$);
    \draw (Mo_left_bottom) rectangle (Mo_right_bottom);
    \node[align=center] at ($(Mo)+(0,-\widthMo/2-0.5)$) (Mo_text) {非兴趣掩膜及其$\alpha$通道\\$[M_{o},M_{o}^{\alpha}]$};
    %为Mo的方格涂上颜色
        %(0,0)
    \fill[GanYu_midblue!80] ($(Mo_left_bottom)+(\widthMo/3*0, \widthMo/3*0)$) rectangle ($(Mo_left_bottom)+(\widthMo/3*0+\widthMo/3, \widthMo/3*0+\widthMo/3)$);
        %(1,0)
    \fill[GanYu_midblue!76] ($(Mo_left_bottom)+(\widthMo/3*0, \widthMo/3*1)$) rectangle ($(Mo_left_bottom)+(\widthMo/3*0+\widthMo/3, \widthMo/3*1+\widthMo/3)$);
        %(2,0)
    \fill[NekasuBlue!78] ($(Mo_left_bottom)+(\widthMo/3*0, \widthMo/3*2)$) rectangle ($(Mo_left_bottom)+(\widthMo/3*0+\widthMo/3, \widthMo/3*2+\widthMo/3)$);
        %(1,0)
    \fill[NekasuBlue!65] ($(Mo_left_bottom)+(\widthMo/3*1, \widthMo/3*0)$) rectangle ($(Mo_left_bottom)+(\widthMo/3*1+\widthMo/3, \widthMo/3*0+\widthMo/3)$);
        %(1,1)
    \fill[GanYu_midblue!88] ($(Mo_left_bottom)+(\widthMo/3*1, \widthMo/3*1)$) rectangle ($(Mo_left_bottom)+(\widthMo/3*1+\widthMo/3, \widthMo/3*1+\widthMo/3)$);
        %(1,2)
    \fill[GanYu_lightblue!91] ($(Mo_left_bottom)+(\widthMo/3*1, \widthMo/3*2)$) rectangle ($(Mo_left_bottom)+(\widthMo/3*1+\widthMo/3, \widthMo/3*2+\widthMo/3)$);
        %(2,0)
    \fill[GanYu_midblue!70] ($(Mo_left_bottom)+(\widthMo/3*2, \widthMo/3*0)$) rectangle ($(Mo_left_bottom)+(\widthMo/3*2+\widthMo/3, \widthMo/3*0+\widthMo/3)$);
        %(2,1)
    \fill[GanYu_midblue!64] ($(Mo_left_bottom)+(\widthMo/3*2, \widthMo/3*1)$) rectangle ($(Mo_left_bottom)+(\widthMo/3*2+\widthMo/3, \widthMo/3*1+\widthMo/3)$);
        %(2,2)
    \fill[NekasuBlue!71] ($(Mo_left_bottom)+(\widthMo/3*2, \widthMo/3*2)$) rectangle ($(Mo_left_bottom)+(\widthMo/3*2+\widthMo/3, \widthMo/3*2+\widthMo/3)$);
    % 为Mo划上方格, 表示一个矩阵
    \foreach \y in {0, \widthMo/3, 2*\widthMo/3}{
        \draw[black,thin] ($(Mo_left_bottom)+(0,\y)$) -- ($(Mo_left_bottom)+(\widthMo,\y)$);
    }
    \foreach \x in {0, \widthMo/3, 2*\widthMo/3}{
        \draw[black, thin] ($(Mo_left_bottom)+(\x,0)$) -- ($(Mo_left_bottom)+(\x,\widthMo)$);
    }

    %创建一个oplus
    \coordinate (oplus) at ($(0,0)+(7, -1)$);
    \node at (oplus) (oplus) {\large $\oplus$};

    %创建一个方形, 本阶段输出: Ofinal
    \coordinate (Ofinal) at ($(0,0)+(9, -1)$);
    \def\widthOfinal{1.2};
    \coordinate (Ofinal_left_bottOfinal) at ($(Ofinal)+(-\widthOfinal/2,-\widthOfinal/2)$);
    \coordinate (Ofinal_right_bottOfinal) at ($(Ofinal)+(\widthOfinal/2,\widthOfinal/2)$);
    \coordinate (Ofinal_left) at ($(Ofinal)+(-\widthOfinal/2-0.1,0)$);
    \coordinate (Ofinal_right) at ($(Ofinal)+(\widthOfinal/2+0.1,0)$);
    \coordinate (Ofinal_upper) at ($(Ofinal)+(0,\widthOfinal/2+0.1)$);
    \coordinate (Ofinal_bottOfinal) at ($(Ofinal)+(0,-\widthOfinal/2-0.1)$);
    \draw[fill=GanYu_ricewhite!30!GanYu_lightblue!90] (Ofinal_left_bottOfinal) rectangle (Ofinal_right_bottOfinal);
    \node[align=center] at ($(Ofinal)+(0,-\widthOfinal/2-0.5)$) (Ofinal_text) {最终风格迁移结果\\$O_{\rm{final}}$};

    % 绘制箭头
    % Om上的出度箭头1:Om->odot1
    \draw[->, thick] (Om_right) -- (odot1.west);
    
    % Oo上的出度箭头1:Oo->odot2
    \draw[->, thick] (Oo_right) -- (odot2.west);

    % Mi上的出度箭头1:Mi->odot1
    \draw[->, thick] (Mi_text.south) -- (odot1.north);

    % Mo上的出度箭头1:Mo->odot1
    \draw[->, thick] (Mo_upper) -- (odot2.south);

    % odot1的出度箭头1:odot1->oplus
    \draw[->, thick] (odot1.east) -| (oplus.north);
    
    % odot2的出度箭头1:odot2->oplus
    \draw[->, thick] (odot2.east) -| (oplus.south);

    %oplus的出度箭头1:oplus->Ofinal
    \draw[->, thick] (oplus.east) -- (Ofinal_left);

\end{tikzpicture}

\end{document}
