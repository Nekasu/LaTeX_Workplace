\documentclass[tikz,convert={outext=.svg}]{standalone}
% \documentclass{paper}
\usepackage{D:/DevData/LaTeX_Workplace/2_macro_definition/network_tikz_package}
\RequirePackage[CJKnumber=true]{xeCJK}
\begin{document}

\begin{tikzpicture}
    % \draw[help lines] (-3,-10) grid (20,10);
        % Use the new command to draw a single rounded rectangle
    % \rSquare{0.6}{NekasuBlue}{20}{5pt}{0}{0}{rs1}

    %表示图像中两个元素之间的水平偏移
    \def\offX{3}
    %表示图像中两个元素之间的垂直偏移
    \def\offY{2}
    
    %创建一个方形, 本阶段输入1:Rceo
    \coordinate (Rceo) at ($(0,0)+(0, \offY)$);
    \def\widthRceo{1.2};
    \coordinate (Rceo_left_bottom) at ($(Rceo)+(-\widthRceo/2,-\widthRceo/2)$);
    \coordinate (Rceo_right_bottom) at ($(Rceo)+(\widthRceo/2,\widthRceo/2)$);
    \coordinate (Rceo_left) at ($(Rceo)+(-\widthRceo/2-0.1,0)$);
    \coordinate (Rceo_right) at ($(Rceo)+(\widthRceo/2+0.1,0)$);
    \coordinate (Rceo_upper) at ($(Rceo)+(0,\widthRceo/2+0.1)$);
    \coordinate (Rceo_bottom) at ($(Rceo)+(0,-\widthRceo/2-0.1)$);
    \draw[fill=GanYu_ricewhite!30] (Rceo_left_bottom) rectangle (Rceo_right_bottom);
    \node[align=center] at ($(Rceo)+(0,-\widthRceo/2-0.5)$) (Rceo_text) {内容非兴趣对象\\$R_{ce\_o}$};

    %创建一个方形, 本阶段输入2:Sb
    \coordinate (Sb) at ($(0,0)+(0, -\offY)$);
    \def\widthSb{1.2};
    \coordinate (Sb_left_bottom) at ($(Sb)+(-\widthSb/2,-\widthSb/2)$);
    \coordinate (Sb_right_bottom) at ($(Sb)+(\widthSb/2,\widthSb/2)$);
    \coordinate (Sb_left) at ($(Sb)+(-\widthSb/2-0.1,0)$);
    \coordinate (Sb_right) at ($(Sb)+(\widthSb/2+0.1,0)$);
    \coordinate (Sb_upper) at ($(Sb)+(0,\widthSb/2+0.1)$);
    \coordinate (Sb_bottom) at ($(Sb)+(0,-\widthSb/2-0.1)$);
    \draw[fill=GanYu_ricewhite!30] (Sb_left_bottom) rectangle (Sb_right_bottom);
    \node[align=center] at ($(Sb)+(0,-\widthSb/2-0.5)$) (Sb_text) {风格背景对象\\$S_b$};

    %创建一个方形, Rb
    \coordinate (Rb) at ($(Rceo)+(\offX, 0)$);
    \def\widthRb{1.2};
    \coordinate (Rb_left_bottom) at ($(Rb)+(-\widthRb/2,-\widthRb/2)$);
    \coordinate (Rb_right_bottom) at ($(Rb)+(\widthRb/2,\widthRb/2)$);
    \coordinate (Rb_left) at ($(Rb)+(-\widthRb/2-0.1,0)$);
    \coordinate (Rb_right) at ($(Rb)+(\widthRb/2+0.1,0)$);
    \coordinate (Rb_upper) at ($(Rb)+(0,\widthRb/2+0.1)$);
    \coordinate (Rb_bottom) at ($(Rb)+(0,-\widthRb/2-0.1)$);
    \draw[fill=Nahida_darkgreen!40] (Rb_left_bottom) rectangle (Rb_right_bottom);
    \node[align=center] at ($(Rb)+(0,-\widthRb/2-0.5)$) (Rb_text) {非兴趣模糊对象\\$R_{b}$};

    %创建一个方形, Sbg
    \coordinate (Sbg) at ($(Sb)+(\offX, 0)$);
    \def\widthSbg{1.2};
    \coordinate (Sbg_left_bottom) at ($(Sbg)+(-\widthSbg/2,-\widthSbg/2)$);
    \coordinate (Sbg_right_bottom) at ($(Sbg)+(\widthSbg/2,\widthSbg/2)$);
    \coordinate (Sbg_left) at ($(Sbg)+(-\widthSbg/2-0.1,0)$);
    \coordinate (Sbg_right) at ($(Sbg)+(\widthSbg/2+0.1,0)$);
    \coordinate (Sbg_upper) at ($(Sbg)+(0,\widthSbg/2+0.1)$);
    \coordinate (Sbg_bottom) at ($(Sbg)+(0,-\widthSbg/2-0.1)$);
    \draw[fill=Nahida_lightgreen!30] (Sbg_left_bottom) rectangle (Sbg_right_bottom);
    \node[align=center] at ($(Sbg)+(0,-\widthSbg/2-0.5)$) (Sbg_text) {风格背景对象特征\\$S_{bg}$};

    %创建一个矩形, AdaIN
    \coordinate (AdaIN) at ($(Sbg)+(\offX, \offY)$);
    \def\lengthAdaIN{1.4};
    \def\widthAdaIN{2};
    \coordinate (AdaIN_left_bottom) at ($(AdaIN)+(-\widthAdaIN/2,-\lengthAdaIN/2)$);
    \coordinate (AdaIN_right_bottom) at ($(AdaIN)+(\widthAdaIN/2,\lengthAdaIN/2)$);
    \coordinate (AdaIN_left) at ($(AdaIN)+(-\widthAdaIN/2-0.1,0)$);
    \coordinate (AdaIN_right) at ($(AdaIN)+(\widthAdaIN/2+0.1,0)$);
    \coordinate (AdaIN_upper) at ($(AdaIN)+(0,\lengthAdaIN/2+0.1)$);
    \coordinate (AdaIN_bottom) at ($(AdaIN)+(0,-\lengthAdaIN/2-0.1)$);
    \draw[fill=GanYu_midblue!40!blue!30, rounded corners=10pt] (AdaIN_left_bottom) rectangle (AdaIN_right_bottom);
    \node[align=center] at ($(AdaIN)$) (AdaIN_text) {自适应\\实例归一化\\$AdaIN$};


    %创建一个方形, Oo
    \coordinate (Oo) at ($(AdaIN)+(\offX, 0)$);
    \def\widthOo{1.2};
    \coordinate (Oo_left_bottom) at ($(Oo)+(-\widthOo/2,-\widthOo/2)$);
    \coordinate (Oo_right_bottom) at ($(Oo)+(\widthOo/2,\widthOo/2)$);
    \coordinate (Oo_left) at ($(Oo)+(-\widthOo/2-0.1,0)$);
    \coordinate (Oo_right) at ($(Oo)+(\widthOo/2+0.1,0)$);
    \coordinate (Oo_upper) at ($(Oo)+(0,\widthOo/2+0.1)$);
    \coordinate (Oo_bottom) at ($(Oo)+(0,-\widthOo/2-0.1)$);
    \draw[fill=Nahida_darkgreen!50!green!30] (Oo_left_bottom) rectangle (Oo_right_bottom);
    \node[align=center] at ($(Oo)+(0,-\widthOo/2-0.5)$) (Oo_text) {风格化背景对象\\$O_{o}$};

    
    % 绘制箭头
    % Sceo上的出度箭头1:Sceo->Rb
    \draw[->, thick] (Rceo_right) -- (Rb_left) node[midway, above] {高斯模糊} ;

    % Sm上的出度箭头1:Sm->Sbg
    \draw[->, thick] (Sb_right) -- (Sbg_left) node[midway, above] {VGG19};

    % Rb 上的出度箭头1:Rb_AdaIN
    \draw[->, thick] (Rb_right) -- ($(Rb_right)+(0.6,0)$) |- (AdaIN_left);

    % Sbg 上的出度箭头1:Sbg_AdaIN
    \draw[->, thick] (Sbg_right) -- ($(Sbg_right)+(0.6,0)$) |- (AdaIN_left);

    % AdaIN 上的出度箭头1: AdaIN -> Oo
    \draw[->, thick] (AdaIN_right) -- (Oo_left);
\end{tikzpicture}

\end{document}
