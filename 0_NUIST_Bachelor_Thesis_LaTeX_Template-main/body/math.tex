\section{排版数学公式}

\LaTeX 强大之处就在于其有很强的数学公式排版能力,如果再加上amsmath宏包,更是如虎添翼。其中\$  \$表示行内公式,如\verb|$a^2+b^2=c^2$|会生成$a^2+b^2=c^2$,而\$\$  \$\$表示行间公式,其实这还不算什么,最重要的是即使处理高度比较高的行间公式时,\LaTeX 也会自动处理而不会使文章中的行距突兀地增大,如$x_{1,2}= \frac{-b\pm \sqrt{b^2-4ac}}{2a}$,而命令\verb|$$a^2+b^2=c^2$$|,会产生$$a^2+b^2=c^2$$
上面的就是一个行间公式。

\begin{equation}\label{fomula}
    \begin{cases}
        \dfrac{du}{dt}=-\dfrac{\partial \phi}{\partial x}+fv \\[1.5ex]
        \dfrac{dv}{dt}=-\dfrac{\partial \phi}{\partial y}-fu \\[1.5ex]
        \dfrac{\partial \phi}{\partial p}=-\dfrac{1}{\rho}   \\[1.5ex]
        p= \rho RT                                           \\[1.5ex]
        \dfrac{\partial u}{\partial x}+\dfrac{\partial v}{\partial y}+
        \dfrac{\partial \omega}{\partial p}=0                \\[1.5ex]
        \dfrac{\partial T}{\partial t}+\overrightarrow{V}\times \nabla_pT-(\Gamma_d-\Gamma)\omega=\dfrac{Q}{c_p}
    \end{cases}
    \text{。}
\end{equation}

式~\ref{fomula}~中为气象上常用的大气运动基本方程组,而且这是一个带编号的公式。

其他公式示例

多行公式使用\&对齐
\begin{equation}
    \begin{aligned}
        A   & =1   \\
        BBB & =222
    \end{aligned}
\end{equation}

无序号公式 equation*
\begin{equation*}
    x=1
\end{equation*}

公式排版就简单介绍到这里吧,因为对笔者所学的专业来说,论文中很少涉及这方面内容,当然个别研究领域可能会出现许多公式,那就是比较高深的领域了,一般是做动力机理或数值模拟研究时常会用到公式,如果有兴趣可以参见\cite{x1}。
