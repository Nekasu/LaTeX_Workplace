\subsection{基于像素迭代的风格迁移技术}

DeepDream\cite{mordvintsevInceptionismGoingDeeper2015}第一个将卷积神经网络(Convolutional Neural Networks,CNN)提取出的结果进行反向重建,从而试图得到一张具有艺术风格的图像\cite{jingNeuralStyleTransfer2020},通过这种方式,DeepGream将深度学习与图像生成结合起来,这种“利用神经网络进行特征提取,再使用其他方法进行”为后来利用神经网络进行风格迁移打下了一定的基础。

在DeepDream之后,Gatys等人\cite{gatysImageStyleTransfer2016}于2016年再次将深度学习与风格迁移相结合,作出了突破性的成果,风格迁移的效果达到了一个新的高度%〔此处需要插入图片:Gatys等人风格迁移的成果〕。
在这之后,神经风格迁移的研究数量逐渐增长,目前正蓬勃发展。


\subsubsection{基于参数化模型的风格迁移}

Gatys等人\cite{gatysImageStyleTransfer2016}首先将深度学习与风格迁移结合,引领了基于参数化模型的风格迁移的热潮。
Gatys等人在于2016年发表的文献\cite{gatysImageStyleTransfer2016}中,将CNN与风格迁移结合,其主要工作可以分为两个部分:基于CNN的参数化纹理模拟以及基于反衍的图像重建过程\cite{gatysImageStyleTransfer2016}。基于CNN的参数化纹理模拟的思想来自于Gatys等人的一个发现,即一个使用足够数据训练的CNN可以提取进行跨数据集的图像特征提取\cite{gatysImageStyleTransfer2016}。在风格迁移领域中,对于一张照片而言,一个预训练的CNN网络能够提取其中的内容信息;对于一张具有特定风格的艺术图像,该CNN能够提取其中的风格信息。Gatys等人以此发现为基础,提出了一种基于像素迭代的风格迁移方法:定义一个损失函数,该损失函数以内容图像的高层特征与正在进行风格化的图像的高层特征之间的差异、以及风格图像与正在风格化的图像的高层特征之间的差异为主要内容。在该损失函数的指导下,将一张噪声图像认定为正在风格化的图像,对该噪声图像进行优化,直到损失函数达到最小值,以获得最终的风格化图像。损失函数的具体形式如下所示:
\begin{equation}
    \label{Gatys_total_loss}
    L_{total}=\alpha L_{content}+ \beta L_{style}
\end{equation}
其中,$L_{total}$是总体的损失函数,$L_{content}$是内容损失,代表正在风格化的图像与内容图像之间的内容差异程度,$L_{style}$是风格损失,代表正在风格化的图像与风格图像之间的风格差异程度,$\alpha$和$\beta$是超参数,用于控制生成的风格化图像与内容图像、风格图像之间的相似程度。

具体来说,内容损失函数$L_{content}$可以写成如下形式:
\begin{equation}
    L_{content}\left(\vec{p},\vec{x},l\right)=\frac{1}{2} \sum_{i,j} \left(F_{ij}^l-P_{ij}^l\right)^2
\end{equation}
其中,$\vec{p}$是输入的内容图像的展开,以向量的形式输入网络;$\vec{x}$是正在风格化的图像,同样的,需要对其进行展开成响亮的处理;$l$代表网络中的第$l$层,$F^l$为正在进行风格迁移的图像经过网络中第$l$层所生成的所有特征图的集合,$F_{ij}^l$表示上述风格化中的图像在神经网络第$l$层生成的特征图集合$F^l$的第$i$个特征图的展开向量中位于$j$位置的值;同理,$P^l$指的是输入网络的内容图像经过神经网络中的第$l$层处理后生成的所有内容特征图的集合,$P_{ij}^l$为上述内容特征图集合中的第$i$张内容特征图的展开向量中位于$j$位置的值。上述公式的含义在于逐像素计算风格化中的图像的特征图与内容图像的对应的特征图的差值并求和,利用梯度下降等方式对齐进行处理,使其最终稳定在一个较小的值,此时视为优化成功。

另一方面,在给出风格迁移函数$L_{style}$的具体表达式之前,需要首先介绍其核心——Gram矩阵。为了提取输入图像的风格特征,Gatys等人\cite{gatysImageStyleTransfer2016}使用了一个旨在捕获纹理信息的特征空间,该特征空间可以建立在网络任意卷积层的输出之上。该特征空间有不同卷积层的特征图之间的相关性构成,而上述相关性用第$l$层中第$i$张以及第$j$张向量化的特征图的内积表示,该内积就是Gram矩阵,记作$G^i \in R^{N_l\times N_l}$,Gram矩阵有公式如下:
\begin{equation}
    G_{ij}^l=\sum_k F_{ik}^lF_{jk}^L
\end{equation}

在此基础上,Gatys等人通过利用梯度下降对噪声图像进行处理,并以最小化原始图像的Gram矩阵与风格图像的Gram矩阵之间的均方距离为目标。网络中第$l$层在风格损失函数方面对$L_{style}$的贡献如下
\begin{equation}
    \label{Gatys_style_loss_l}
    E_l = \frac{1}{4N_l^2M_l^2}\sum_{ij}\left(G_{ij}^l-A_{ij}^l\right)^2
\end{equation}
其中,$N_l$为网络中一个卷积层中卷积核的个数,也即该层网络能够生成的特征图的个数,$M_l$是特征图包含的像素的个数,在数值上等于特征图的长度与宽度的乘积,$A_{ij}^l$和$G_{ij}^l$分别表示网络第$l$层中第$i$张向量化特征图上位于$j$位置的像素值。公式\ref{Gatys_style_loss_l}仅计算了网络单层对风格损失$L_{style}$的影响,将所有层中的损失带权相加即为$L_{style}$,其具体形式如下:
\begin{equation}
    L_{style}\left(\vec{a},\vec{x}\right)=\sum_{l=0}^L \omega_l E_l
\end{equation}  
其中,$w_l$是每一层对于总风格损失函数的影响权重,$L$为网络中卷积层的个数。将式\ref{Gatys_total_loss}进行偏导计算,得到结果$\frac{\partial L_{total}}{\partial \vec{x}}$,该结果可以作为作为优化算法的输入,并指导风格化中的图像进行迭代,以完成风格迁移的效果。

Gatys等人首创的以Gram矩阵为核心的风格迁移方法具有鲜明的优点。与以往的传统风格迁移方法相比,具有突破传统风格迁移效果笔触呆板、变化度少的优秀的效果;同时,突破了传统方法仅能对特定风格进行迁移的缺陷,实现了对自然的纹理以及风格化的纹理进行迁移\cite{jingNeuralStyleTransfer2020},从而获得良好的迁移效果。但是,此方法存在一些明显的缺陷。由于每次风格迁移都是从一张噪声图像开始,因此在批量进行风格迁移时,需要花费大量的时间,并在实时风格迁移方面效果较差。除此之外,Gram矩阵更擅长提取特征图的全局信息,这导致对具有长程对称结构的规则纹理的提取效果不能令人满意\cite{jingNeuralStyleTransfer2020}。同时,与传统风格迁移忽略高层语义信息不同的是,由于Gatys等人的方法仅仅考虑了图像中的高层语义信息,而忽略了其中的低层语义信息,导致了合成的风格化图像中精细结构与细节连贯性方面有所缺陷。

为了解决这个问题,Berger\cite{bergerIncorporatingLongrangeConsistency2016}等人对Gaty等人进行改进,提出了水平垂直像素差异,该方法计算了特征图中位于$(i,j)$位置的像素与位于$(i,j+\delta)$或位于$(i+\delta,j)$的像素之间特征关系,并将之纳入风格损失的考量。通过这种方式,能更有效的对具有对称性的长纹理图案进行迁移,从一定程度上弥补了Gatys等人对于精细结构与长程对称结构的规则纹理的模拟效果的不足。同时,由于使用了Gram矩阵作为风格迁移的核心,所以本方法依旧保留了一部分Gatys等人方法的缺陷,如对细节纹理的模拟不到位。

Risser\cite{risserStableControllableNeural2017}等人针对Gatys等人的方法进行研究,并发现了Gram矩阵用于风格迁移时不稳定的原因:具有不同均值和方差的特征图可能拥有相同的Gram矩阵。基于这个发现,Risser等人将特征图的直方图纳入风格迁移损失函数的考量,进一步优化了基于Gram矩阵的风格迁移方法。其优点在于能够生成更加稳定的风格图像;但由于引入了特征图的直方图,导致该方法的计算更加复杂,风格迁移过程耗时更长,在进行批量迁移时效率更低。同时,由于Risser等人仅考虑了风格化过程中生成图像的稳定性问题,不能对纹理精细化描述的问题依旧存在。

上述基于Gram矩阵的神经风格迁移效果存在一些共有的缺陷。由于卷积神经网络丢失了一些图像中的低层信息,导致对于具有规则形状的物体(如人工制造的物件)的迁移结果中往往存在一些不可忽视的扭曲现象。同时,风格化的过程会花费大量的时间,在对大量图像进行风格迁移时需要大量的时间,在诸如视频风格迁移等具有大量图像的任务中效率较低,且生成风格不稳定。

\subsubsection{以MMD及其变体作为损失函数}

Li等人\cite{liDemystifyingNeuralStyle2017}认为神经风格迁移任务可以视作领域自适应任务的一个特殊的变种任务,以此为切入点,Li等人得到了不一样的视角。领域自适应任务基于一个事实,即源数据与目标数据的分布不同,其目的是通过在一个带有标签的源域数据集上进行训练,以得到一个能够预测目标领域数据分布情况的模型。领域自适应任务的一种方法是通过最小化源域和目标域中样本的分布差异,从而实现源域与目标域中样本的匹配,其中最大均值差异(Maximum Mean Discrepancy,MMD)是度量两个域之间差异的常用选择。类比到风格迁移任务中,内容图像可以看作是该领域自适应的源域,而风格化的图像即是目标域。Li等人探究了Gram矩阵在风格迁移中的数学作用,证明了对风格图像与风格化中的图像的Gram矩阵的匹配过程,即公式\ref{Gatys_style_loss_l},本质上与最小化一个具有二次多项式核的MMD相同。因此,Li等人认为,最小化具有其他核函数(如线性核、多项式核、高斯核)的MMD可能会在风格迁移领域中具有一定的作用。Li等人的主要贡献在于从理论方面探寻并发现Gatys等人方法的原理,使其在原理方面更加清晰。

\subsubsection{基于非参数化模型的风格迁移}

使用马尔科夫随机场(Markov Random Fields,MRF)进行纹理模拟是传统风格迁移领域中一个较为常见的方法\cite{ChenHongJiYuYangBenXueXiDeXiaoXiangHuaZiDongShengChengSuanFa2003,liMarkovRandomField1994,crossMarkovRandomField1983,chellappaTextureSynthesisCompression1985,bennettMultispectralRandomField1998},Li和Wand等人\cite{liCombiningMarkovRandom2016}将MRF与深度卷积神经网络(Deep Concolutional Neueal Networks,dCNN)相结合,提出了非参数化的神经风格风格迁移方法。他们认为,使用Gram矩阵的参数化风格迁移方法仅考虑了像素和像素之间的差异,没有从空间层次角度对风格化图像进行约束,从而导致了在生成的图像合理性不足。因此,Li和Wand等人将Gram矩阵替换为MRF正则化器,并引入了一个新的损失函数:
\begin{equation}
    L_s=\sum_{l\in {l_s}}\sum_{i=1}^m \mid\mid\Psi_i(F^l(I))-\Psi_{NN(i)}(F^l(I_s))\mid\mid^2
\end{equation}其中,$\Psi(F^l(I))$是特征图$F^(I)$所有局部块的集合;$\Psi_i(F^l(I))$为特征图所有局部块集合中的第$i$个;$\Psi_{NN(i)}$是风格化图像$I$中与第$i$个局部块风格最相似的风格块,可以通过计算风格图像中所有风格块的归一化关系从而得到上述最佳匹配的风格化块$\Psi_{NN(i)}$;$m$为局部块的总数。Li和Wand等人的方法增强了Gatys等人风格迁移的效果,使得风格化图像中物体的结构更具合理性,可以更好的保留原图的精细结构,并在合成真实照片方面取得了较大的进步。但同时,如果内容与风格图像在结构上存在较大差异时,局部块与风格块的匹配度可能不高,图像块无法正确匹配,最终导致该方法生成的风格图像的效果较差。


\subsubsection{基于Diffusion的风格迁移}

Diffusion模型是一种生成模型,最早由Ho等人\cite{hoDenoisingDiffusionProbabilistic2020}给出了详细的数学证明、推导与可运行的代码。Diffusion模型可以从噪声中生成目标数据样本。它包括两个过程:前向过程(forward process)和逆向过程(reverse process),其中前向过程又称为扩散过程(diffusion process)。前向过程是加噪的过程,前向过程中图像$ x_t $只和上一时刻的 $x_ {t-1}$ 有关, 该过程可以视为马尔科夫过程, 满足: 
\begin{equation}
    q (x_ {1:T}|x_0) = \prod_ {t = 1}^ {T}q (x_t|x_ {t-1}) q (x_t|x_ {t-1}) = N (x_t, \sqrt {1-\beta_t}x_ {t-1},\beta_t I)
\end{equation} 
其中不同t的 $\beta_t$ 是预先定义且逐渐衰减的,并满足 $\beta_1<\beta_2<...<\beta_T$。逆向过程是去噪的过程,如果得到逆向过程$ q (x_ {t-1}|x_ {t})$ ,就可以通过随机噪声 $x_T$ 逐步还原出一张图像。

Hamazaspyan与Navasardyan\cite{hamazaspyanDiffusionEnhancedPatchMatchFramework2023}将diffusion模型与风格迁移任务相结合,提出了扩散增强的块匹配(Diffusion-Enhanced PatchMatch,DEPM)模型。该模型利用Stable Diffusion来捕获高级风格特征,同时保留原始图像的细粒度纹理细节。DEPM允许在推理过程中转移任意样式,而无需任何微调或预训练,从而使过程更加灵活和高效。

Zhang等人\cite{zhangInversionBasedStyleTransfer2023}在Diffusion模型的基础上提出了具有全新思想的风格迁移——将学习各艺术风格中的隐含文字标签作为风格迁移核心。该方法的主要思想是将艺术画中的风格看作一幅画的可学习的文本描述,并根据风格标签指导Diffusion模型生成图像。在具体实现方面,Zhang等人利用所提出的反衍风格迁移方法(Inversion-Based Style Transfer Method,InST)高效准确的学习图像相关信息。具体来说,使用条件生成模型学习图像和文本之间的对应关系,从而获得图像嵌入;在图像嵌入的基础上,利用注意力导向的反转模块接收到图像嵌入,并利用注意力机制来生成对应的文本嵌入。该模块会关注图像嵌入中的不同特征,例如语义、材质、对象形状、笔触和颜色等,最终得到对应的文本嵌入,及文本标签。利用文本标签即可指导Diffusion模型进行风格迁移。该文本标签并不一定可以用自然语言描述出来,是一种只有Diffusion模型可以读懂的对风格进行描述的一串字符或者一个令牌(token)。该文本标签一经学习,Diffusion即可固定该标签对应风格,在这个层面上,可以视本方法为基于模型迭代的风格迁移方法。本文最大的特色是可以改变风格化过程中图像的形状,这是以往风格迁移模型所不能的。

Wang等人\cite{wangStyleDiffusionControllableDisentangled2023}利用Diffusion模型实现了图像风格与图像内容的分离。Wang等人构建了一种名为StyleDiffusion的框架,用于实现可控的分离风格转换。该框架基于Diffusion模型,通过扩散过程分别移除图像中的风格信息和内容信息,并通过协调样式重建先验的基于CLIP的样式解缠损失,实现了内容和风格的完全解缠。框架由三个关键组成部分:基于扩散的风格去除模块、基于扩散的风格转移模块以及与风格重建先验协调的基于CLIP的风格解缠损失。实验证明,该框架能够生成高质量的风格转换结果,并且相对于其他方法,更好地考虑了内容和风格之间的关系。与之前的方法相比,本文方法通过扩散模型完全解耦了内容(C)和风格(S),从而更好地考虑了它们之间的关系。这样一来,风格转换结果更加自然和和谐(尤其对于具有挑战性的风格,如立体派和油画)。Wang等人的方法通过扩散模型和基于CLIP的风格解耦损失,实现了对风格转换过程的精确控制。通过调整参数,可以灵活地控制风格去除的程度和内容与风格的解耦程度,从而得到更加理想的风格迁移效果。同时,该模式具有较高的可解释性和可扩展性。通过引入扩散模型和基于CLIP的风格解耦损失,风格迁移过程更具可解释性。此外,该方法还可以应用于其他图像转换或操作任务,具有一定的扩展性。

Lu等人\cite{luSpecialistDiffusionPlugandPlay2023}试图利用Diffusion模型解决如何通过少量图像样本对预训练的扩散模型进行微调,以学习任何未见过的风格的问题。Lu等人提出了一种名为"Specialist Diffusion"的方法,它可以通过对预训练的扩散模型进行微调来学习任何未见过的风格。通过仅使用少量图像(例如少于10张),便可以对预训练的扩散模型进行微调,使其能够以指定风格生成任意对象的高质量图像。为了实现这种极低样本微调,Lu等人提出了一套新颖的微调技术,包括文本到图像的定制数据增强、内容损失以促进内容和风格的解耦,以及只关注少数时间步骤的稀疏更新。"Specialist Diffusion"方法可以与现有的扩散模型和其他个性化技术无缝集成,并在学习高度复杂的风格时,以超级高效的微调效果胜过最新的少样本个性化扩散模型。此外,"Specialist Diffusion"可以与反转方法相结合,进一步提高性能,甚至在非常不寻常的风格上也能取得成功。但同时,该方法存在一定的局限性。首先,虽然本方法可以通过少量图像进行微调来学习未见过的风格,但是对于一些高度特定和不寻常的风格,仍然可能存在学习不充分的情况。其次,本方法在样本效率上取得了很好的效果,但对于某些复杂的或接近训练数据分布的内容,生成的结果可能不尽如人意。最后,本方法的性能受到预训练模型的限制,如果预训练模型的质量不高,可能会影响到微调后的结果。

由于卷积神经网络(CNNs)的局部性,提取和维护输入图像的全局信息变得困难,传统的神经风格迁移方法存在内容偏倚的问题。为了解决这个问题,Deng等人\cite{dengStyTr2ImageStyle2022}提出了一种基于Transformer的方法,称为StyTr2,该方法考虑了输入图像的长距离依赖关系。与其他视觉任务的视觉Transformer不同,StyTr2 包含两个不同的Transformer编码器,分别用于生成内容和风格的特定领域序列。在编码器之后,采用多层Transformer解码器根据风格序列对内容序列进行风格化。同时,Deng等人还分析了现有位置编码方法的不足,并提出了适用于图像风格迁移任务的内容感知位置编码(CAPE),它具有尺度不变性,更适合图像风格迁移任务。但是该方法存在一定的局限性。1. 对于与训练示例分辨率不同的输入,学习方法可能会出现垂直轨迹伪影的问题。2.传统的基于CNN的方法在重复迭代的过程中可能会导致生成的内容结构变得模糊,尽管该方法可以减轻这个问题,但仍存在一定程度的内容泄漏。3.本文中使用的Transformer架构相对于传统的CNN方法可能会增加计算复杂度,导致算法的运行时间较长。