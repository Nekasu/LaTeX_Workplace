\section{评价标准}

风格迁移领域的评价可以分为定性评价和定量评价。其中,定性评价是有关评价一张风格化图像的艺术性的标准,该标准多依赖于观察者的审美判断,评价的结果往往与观察者的年龄、职业等诸多自身因素相关,因此难以进行客观的评价;定量评价主要侧重于对风格迁移模型的性能、精确度定方面进行评价。

\subsection{定性标准}

由于对于模型生成的风格化图像的定性评价与观察的如年龄、职业等诸多自身因素相关,因此目前没有一个客观的评价标准足以对风格化图像的艺术风格效果进行评价。对于生成图像的艺术性判断,大多是向社会人群发放问卷,根据问卷结果进行评估与评价。

\subsection{定量评价}

定量评价主要关注一下5个指标:生成单张图像所需时间、单个模型所需的训练时间、风格化图像与内容图片的平均损失、训练过程中的损失变化、模型的可拓展性。

生成单张图像所需时间的长短反应了模型是否能够进行实时风格化;风格化图像与内容图片的平均损失可以用于衡量损失函数的优化方式;训练过程中的损失变化则反应了模型的收敛情况。
