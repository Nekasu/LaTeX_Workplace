\section{Related Work}

\textbf{神经风格迁移} 使用神经网络作为核心, 以风格图像与内容图像作为输入, 获得具有内容图像内容信息与风格图像风格信息的风格化图像的过程, 被称作神经风格迁移. 神经风格迁移借助如 CNN\cite{tammina2019transfer} 等神经网络具有提取全局或局部风格信息的特点, 弥补了传统风格迁移需要人工建模模拟风格纹理的缺陷. To our best Knowledge, 神经风格迁移工作最早由 Gatys等人于2016年\cite{gatys2016image}提出. Gatys等人\cite{gatys2016image} 将噪声图的像素值看作神经网络中的参数, 在以 VGG\cite{simonyan2014very} 为核心的损失函数指导下, 优化噪声图最终得到了风格化图像, 开辟了神经风格迁移的先河. 

但由于优化对象是噪声图像中的像素值, 所以 Gatys等人\cite{gatys2016image} 的方法在处理大批量风格迁移任务时需要分别一次每一个风格-内容图像对, 导致迁移速度缓慢, 效率难以令人满意. 同时, 由于每次生成风格化图像需要等待较长的时间, 所以无法做到实时风格迁移. Ulyanov\cite{ulyanov2016texture} 与 Johnson\cite{johnson2016perceptual} 几乎同时且独立提出了实时风格迁移成果, 且二人的实现思路类似: 参数化特定作品的风格信息, 并用反向传播算法固化在前馈神经网络中, 而不需要在噪声图像上进行多次迭代优化. 
使用马尔科夫随机场也是早期为了实现实时风格迁移的尝试. Li等人\cite{li2016precomputed} 用马尔科夫随机场进一步优化了他们之前的一项工作\cite{li2016combining}, 他们通过对抗训练得到了一个马尔科夫前馈网络, 以解决效率问题.
在这之后, 还有以变分自编码器\cite{zhang2023caster}、生成式对抗网络(Generative Adversarial Network, GAN)\cite{zhu2017unpaired,karras2019style,Men_2022_CVPR}为基础的诸多方法为了提升风格迁移速度而努力.
尽管他们的方法大降低了风格迁移的耗时, 但只能迁移预先训练好的一种风格, 灵活性较差. 

随着风格迁移的进一步发展, 研究者们不再满足于迁移风格数量与效率之间的权衡, 开始追求高效且可以迁移多种风格的方法. 
据我们所知, Dumoulin等人\cite{dumoulin2016learned}第一个在数量-效率权衡之间做出了突破, 他们在保证风格迁移效率的同时, 将可迁移的风格数量从一个扩展至多个. 具体来说, 他们发现对归一化参数进行缩放或变换足以适应特定的风格, 因此提出了 Conditional Instance Normalization(CIN)方法. 通过该方法调整网络参数, 可以完成多种类似但不同的风格迁移任务. 
Chen等人\cite{chen2017stylebank}不同于Dumoulin\cite{dumoulin2016learned}通过公式调整网络参数的方式, 提出了名为StyleBank的网络参数替换技术. 他们认为网络中处理内容信息的部分可以是相同的, 仅需要替换处理风格图像的参数部分即可. 所以他们将训练好的参数保存在一个 "Bank" 中, 当模型被要求迁移某种风格时, 网络将从 "Bank" 中读取对应参数层并加载入网络中. 通过这种参数替换方式, 在保证效率的前提下增加了可迁移风格种类. 
这两种方法作为Per-Model Multi-style方法的典型方法, 分别具有不同的缺陷: Dumoulin等人\cite{dumoulin2016learned}方法以迁移质量为代价, 而Chen等人\cite{chen2017stylebank}模型的参数量会随着可迁移种类的增加而增加, 造成存储与使用困难.

发展到现在, 大部分风格迁移工作均可满足实时且任意(Arbitrary and Real-time)的要求. 按实现方式的不同, 任意且实时风格迁移可以分成五类: 1. 基于全局参数调整; 2. 使用 GANs; 3. 使用注意力机制; 4. 使用预训练大模型; 5. 使用自搭建的网络. 第一个实现任意和同时风格迁移的工作由Huang等人在2017年提出\cite{huang2017arbitrary}, 属于全局参数调整. 受CIN方法\cite{dumoulin2016learned}的启发, Huang等人以自适应实例归一化层(Adaptive Instance Normalization layer, AdaIN)作为风格迁移网络的核心, 该层可以将内容特征的方差和均值与风格特征的均值和方差进行匹配, 从而实现任意的风格迁移. 
Xu等人的工作\cite{xu2021drb}是利用 GAN 实现任意且实时风格迁移的一个典范. 他们提出的DRB-GAN通过动态残差块(Dynamic ResBlocks)统一整合风格编码与迁移网络, 利用风格类感知注意力机制生成动态风格代码, 从而完成任意且实时的风格迁移.
Liu等人于近期提出了一份使用注意力机制辅助风格迁移的有趣工作\cite{liu2304any}. 该工作允许用户从风格图像中交互式地选择特定区域, 并将其对应的风格应用于内容图像中的相同区域, 从而实现个性化的风格迁移效果.
随着如CLIP, Diffusion等模型的发展, 使用这些模型作为辅助的风格迁移工作也逐渐增多. Liu等人\cite{liu2023name}从用户体验的角度出发, 认为使用文字描述代替风格图像是提升用户体验的重要因素. 因此他们提出了基于 CLIP 的 TxST模型用于提取文字描述中的风格信息, 以此作为风格图像特征的代替品, 这使得用户在没有风格图像时, 也能进行风格迁移. Zhang 等人\cite{zhang2023inversion} 基于Diffusion模型提出了一种新的风格迁移方法, 其核心思想是将艺术风格视为可学习的文本标签, 并以此引导扩散模型进行图像生成. 为实现这一目标,他们提出了名为 InST(Inversion-Based Style Transfer)的反演式风格迁移方法, 能够高效且精确地学习图像相关信息, 从而实现风格迁移.
研究者根据自己的想法设计有效的自搭建风格迁移网络.Li 等人\cite{li2023frequency} 不同于以往专注于空间域的方法, 提出从频率域角度解耦图像的内容与风格特征. 基于图像的"频率可分特性", 他们设计了 FreMixer 模块, 通过傅里叶变换将特征图映射至频率域, 完成风格与内容的分离与重构. 该方法不仅有效缓解了其他方法中的特征纠缠问题, 也为风格迁移提供了新的实现思路.

当前风格迁移领域正向着提升迁移质量、提高迁移速度、提升人机交互体验、由单模态转向多模态的方向发展, 但依旧没有考虑到 alpha 通道. 我们的模型, PartAlphaNet, 考虑到现实生活中被设计师广泛使用的带有不透明度通道的图像(如 png 格式的图像), 不再简单舍弃 alpha 通道信息, 而是将 alpha 通道纳入风格迁移考量, 很好的弥补了这一方面的研究空缺.