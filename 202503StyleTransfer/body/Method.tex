\section{Method}

\subsection{Background: Alpha 通道在风格迁移任务中的缺失}

\paragraph{RGB vs. RGBA: 不透明度信息的重要性}
% 引言
在计算机视觉领域, 图像往往以 R、G、B 三个通道的格式存取与处理, 因为这已足以适应绝大多数任务场景. 然而, 当涉及到对图像中不同区域进行精细的显示与隐藏控制, 或者需要调节多张图像间的遮挡与融合程度时, 传统的 RGB 图像便显得力不从心. 
% alpha通道是什么
RGBA 图像在原有的 RGB 图像基础上增加了一个 A 通道, 即 alpha 通道, 该通道用以表示图像每个像素的不透明度. alpha 通道的概念最早可追溯至 Porter 和 Duff 在 1984 年提出的数字图像合成理论, 该理论系统阐述了如何利用 alpha 通道实现图层间的混合与透明度控制\cite{porter1984compositing}. 在实际应用中, alpha 值通常用来指示像素的不透明程度, 其取值范围可以设定为 $[$0, 1$]$ 或 $[$0, 255$]$, 通过直接指定或计算生成, 以反映图像中各部分的显示或隐藏程度. 

在实际应用中, alpha通道 对于实现图像中的局部区域控制具有显著优势. 以图形设计与合成为例, alpha通道 允许设计师指定图像的前景与背景, 并精确调控多图像之间的遮挡关系. 相比之下, 传统的 RGB 图像由于仅包含颜色信息, 在需要实现精细区域控制时往往显得不足.
设想有两幅图像:一幅前景图像为人物肖像, 以 PNG 格式(一种常见的RGBA格式)存储, 其中人物区域完全不透明 (alpha 值接近 1 或 255) , 而背景部分则完全透明 (alpha 值接近 0) ; 另一幅图像为背景图像, 例如一幅风景图. 在传统的 RGB 图像处理中, 仅保留颜色信息的前景图像无法明确区分出真正的前景和需要由背景填充的区域, 因此直接将 RGB 前景叠加到背景上往往会产生不自然的矩形边框和锐利的边缘. 通过提取 PNG 图像中的 alpha 通道, 并将其作为蒙版来使用, 可以仅保留前景图像中不透明区域, 将透明区域交由背景图显示, 从而达到前景与背景自然过渡的效果. 这种方法不仅保留了前景对象的完整形态, 还使背景能够无缝融入, 为图像合成提供了更高的精确性与美观度. 

当前风格迁移工作\cite{johnson2016perceptual,risser2017stable,sanakoyeu2018style,jing2019neural,goodfellow2020generative,li2023frequency,fu2023neural,tang2022few,kwon2024aesfa}大多忽视了图形设计与合成的应用场景. 在网络输入端, 这些方法通常仅接受3通道的 RGB 图像作为输入, 而当输入为 RGBA 图像时, 系统只处理 RGB 部分, 忽略了其中的 alpha 通道. 在网络的处理中, 由于随意抛弃 alpha 信息, 导致生成过程中处理的图像与原始 RGBA 图像之间产生较大差距. 网络可能将这些差距视为图像风格的一部分, 进而在输出图像中引入大量不属于原输入图像的风格信息, 不仅使风格迁移的质量大幅下降, 也无法保留原图的不透明度信息. 这对于实际应用场景——如需要精细控制前景与背景, 以实现无缝图层合成的图形设计工作来说, 是致命的缺陷. 

\paragraph{alpha通道与 vanilla convolution 的不适配性}

RGBA 图像中的 alpha 通道与 RGB 三个通道在信息表达、语义含义及处理需求上存在本质差异. RGB 通道主要承载图像的颜色与亮度信息, 直接参与图像的视觉重建, 而 alpha 通道则用于编码像素的不透明度, 决定图像在合成、遮挡与可视化中的显隐关系, 更多体现为一种控制性的"元信息". 与 RGB 通道的结构性视觉特征不同, alpha 通道的数值变化通常呈现出遮挡块状、边界过渡或局部渐变等规律, 具有独立的数据分布与语义特征. 

然而, 当前工作常使用 vanilla convolution 统一处理输入通道. 作为一个通用模块, vanilla convolution 无法区分 RGB 通道与 alpha通道 之间的本质差异. 因此在训练与推理过程中, vanilla convolution 会错误的将 alpha通道 当作 颜色通道处理, 导致语义污染, 最终导致风格迁移质量下降. 因此, 在处理 RGBA 图像时, 有必要引入对 alpha 通道敏感的建模机制, 以避免信息在特征提取过程中的语义丢失或功能混淆.