\section{Experiment}

此部分中, 我们首先介绍提出的 AlphaImg 数据集. 随后, 我们验证了所提出的 Soft Partial Convolution 在进行具有不透明度信息的风格迁移时的有效性, 并说明其相对于其他先进方法的优越性.

\subsection{Dataset}

为了验证所提出方法在透明图像风格迁移中的有效性, 我们构建了一个名为 AlphaImg 的合成数据集. 该数据集基于 MSCOCO\cite{cocodatasetCOCOCommon} 与 WikiArt\cite{wikiartWikiArtorgVisual}数据集, 通过 Segment Anything Model(SAM)\cite{kirillov2023segment}将整副图像分割并裁剪为多个形状不规则的图像, 并为分割后图像添加 alpha 通道信息, 以 png 格式存储. 
训练集与测试集均由内容图像集与风格图像集组成, 内容图像集以处理后的 MSCOCO 数据为主, 风格图像集以处理后的 WikiArt 数据为主. 训练集包含 50,000 张来自 MSCOCO 且背景透明的内容图像, 以及 10,000 张来自 WikiArt 的风格图像. 测试集中包含 10,000 张来自 MSCOCO 且背景透明的图像, 以及 5,000 张来自 WikiArt 的风格图像. 其中内容图像不透明度仅为0或255, 而风格图像的不透明度则会在0~255之间变化, 以模拟真实情况.

具体来说, 对一张来自 MSCOCO 的内容图像 $C^i$, 我们首先使用 SAM 模型为其创建包含 $n$ 个掩膜的掩膜组 $M_c^i=\{m_c^{i,1}, m_c^{i,2}, \cdots, m_c^{i,n}\}$. 随后将掩膜组 $M_c^i$ 中的每一张掩膜 $m_c^{i,j}$ 当作内容图像 $C^i$ 的 alpha 通道, 以获取 $n$ 张具有透明背景的内容图像. 
% 对于一张来自 WikiArt 的风格图像 $S^i$, 我们同样使用 SAM 模型为其创建一个风格掩膜组, 并按掩膜中白色区域的大小从小到大排序, 形成排序后的风格掩膜组$M_s^i=\{m_s^{i,1}, m_s^{i,2}, \cdots, m_s^{i,n}\}, \sum m_s^{i,1} \le \sum m_s^{i,2} \le \cdots \le \sum m_s^{i,n}$. 随后, 根据白色区域的大小修改像素值, 以模拟风格图像中可能存在的不透明度变化情况. 对于风格图像 $S^i$ 的 第 $j$ 张风格掩膜 $m_s^{i,j}$, 按如下公式修改以获得 Soft Mask $m_s^{'i,j}$:
% \begin{equation}
%     \begin{aligned}
%         m_{s,(x,y)}^{'i,j} = \begin{cases}
%             \frac{j+1}{n} \cdot 255, \quad &\text{if} \quad m_{s,(x,y)}^{i,j}=255, \\
%             0,\quad &\text{if} \quad m_{s,(x,y)}^{i,j}=0 \\
%         \end{cases}
%     \end{aligned}
% \end{equation}
% 对于 Soft Mask 组 $M_s^{'i}=\{m_s^{'i,1},m_s^{'i,2},\cdot,m_s^{'i,n}\}$
对于一张来自 WikiArt 的风格图像 $S^i$, 我们同样使用 SAM 为其生成风格掩膜组. 所有掩膜根据其白色区域的像素数量进行升序排序, 得到排序后的风格掩膜集合:
\begin{equation}
    M_s^i=\{m_s^{i,1}, m_s^{i,2}, \cdots, m_s^{i,n}\}, \sum m_s^{i,1} \le \sum m_s^{i,2} \le \cdots \le \sum m_s^{i,n}
\end{equation}
 
为模拟风格图像中可能存在的透明度分布差异, 我们将每个掩膜 $m_s^{i,j}$ 转换为带有不同灰度级别的 soft mask $m_s^{'i,j}$. 具体地, 对于每个像素位置 $(x, y)$, 其值按如下方式赋值:

\begin{equation}
    \begin{aligned}
        \label{eq:soft-mask}
        m_{s,(x,y)}^{'i,j} = 
        \begin{cases} 
            \frac{j+1}{n} \cdot 255, & \text{if } m_{s,(x,y)}^{i,j} = 255 \\
            0, & \text{if } m_{s,(x,y)}^{i,j} = 0
        \end{cases} 
    \end{aligned}
\end{equation}
该设计使得编号越大的掩膜(即区域面积越大的部分)对应的透明度值越高, 从而体现出一种从局部细节到整体结构的风格强度递进.
为进一步构建一张连续的 alpha 蒙版图, 我们将所有 soft mask 按照如下规则进行叠加合成:

\begin{equation}
    \begin{aligned}
        \label{eq:alpha-composition}
        \tilde{A}^i(x, y) = 
        \begin{cases} m_{s,(x,y)}^{'i,j}, & \text{if } m_{s,(x,y)}^{i,j} = 255 \text{ 且 } (x, y) \text{ 尚未被赋值} \\
        \tilde{A}^i(x, y), & \text{otherwise} \end{cases}
    \end{aligned}
\end{equation}

其中 $\tilde{A}^i$ 表示图像 $S^i$ 的最终透明度图. 我们初始化 $\tilde{A}^i(x, y) = 255$, 然后依序扫描每一个 soft mask. 一旦某个位置 $(x, y)$ 被某个掩膜 $m_s^{i,j}$ 覆盖且未被之前的掩膜赋值, 则用 $m_s^{'i,j}$ 中的值更新之. 该策略可确保不同区域以“后掩膜优先”的方式分配透明度, 从而在一定程度上模拟层叠结构.


\subsection{Style Transfer Results}

在本节中, we compare the results between the proposed Soft Partial Convolution and previous SOTAS, including AdaIN\cite{huang2017arbitrary}, CycleGAN\cite{zhu2017unpaired}, CAST\cite{zhang2022domain}, StyTr2\cite{deng2022stytr2},  QuantArt\cite{huang2023quantart}, ArtBank\cite{zhang2024artbank}, 以及 baseline 工作 AesFA \cite{kwon2024aesfa}. 

% 介绍上述方法, 包括优势、劣势、特色、对风格迁移发展的意义
% AdaIN 将风格迁移领域推广到任意风格迁移, CycleGAN 引入无配对图像映射和循环一致性损失, 二者均是风格迁移领域的经典工作. 

\subsection{Qualitative Comparasion}

由于上述方法没有针对 RGBA 图像进行专门训练, 我们将上述 经典 或 SOTAS 方法 在本文整合的数据集 AlphaStyle 上重新训练, 以进行公平的比较. The qualitative comparisons presented in Figure (picture needed) provide a visual assessment of the outcomes achieved by different style transfer methods.

% 分析本文方法与上述方法比较, 优势在何处

% 总结
与上述方法相比, 我们提出的 Soft Partial Convolution 不仅可以将 alpha 通道纳入风格迁移网络, 也能保留原始输入图像的不透明度关系.
\subsection{User Study}


\subsection{Ablation Study}