\usepackage{tikz}
\usepackage{fp}
\usepackage{ifthen}
\usetikzlibrary{calc}

% 定义颜色
\definecolor{NekasuBlue}{HTML}{6495ED}
% 定义单个圆角矩形命令
\newcommand{\rSquare}[6]{
    % 参数1:圆角矩形的长度与宽度
    % 参数2:填充的颜色
    % 参数3:填充颜色的不透明度
    % 参数4:圆角矩形的弧度
    % 参数5:起始x坐标
    % 参数6:起始y坐标
    \def\hw{#1} % 用第一个变量给\hw变量赋值, 表示圆角矩形的长度与宽度
    \def\fillcolor{#2}
    \def\alphaChannel{#3}
    \def\roundedIndex{#4}
    \def\x{#5}
    \def\y{#6}

    % 定义矩形的四个角的坐标
    \coordinate (left_bottom_A) at (\x, \y);
    \coordinate (right_bottom_B) at (\x+\hw, \y);
    \coordinate (right_upper_C) at (\x+\hw, \hw+\y);
    \coordinate (left_upper_D) at (\x, \hw+\y);

    \draw[fill=\fillcolor!\alphaChannel, ultra thin, rounded corners=\roundedIndex] (left_bottom_A) rectangle (right_upper_C);
}

% 定义多个圆角矩形的命令
\newcommand{\rSquareSet}[9]{
    % 参数1:圆角矩形的长度与宽度
    % 参数2:填充的颜色
    % 参数3:填充颜色的不透明度
    % 参数4:圆角矩形的弧度
    % 参数5:起始x坐标
    % 参数6:起始y坐标
    % 参数7:每个圆角矩形与上一个之间的x,y偏移, 这两个数值相等
    % 参数8:这组圆角矩形的名称
    % 参数9:圆角矩形组与其名称之间的y方向距离
    \def\hww{#1} % 用第一个变量给\hww变量赋值, 表示圆角矩形的长度与宽度
    \def\fillcolorw{#2}
    \def\alphaChannelw{#3}
    \def\roundedIndexw{#4}
    \def\xw{#5}
    \def\yw{#6}
    \def\xxw{#7}
    \def\inText{#8}
    \def\yLen{#9}

    % 绘制一组3个的圆角矩形
    \rSquare{\hww}{\fillcolorw}{\alphaChannelw}{\roundedIndexw}{\xw}{\yw};
    \rSquare{\hww}{\fillcolorw}{\alphaChannelw}{\roundedIndexw}{\xw-\xxw}{\yw-\xxw};
    \rSquare{\hww}{\fillcolorw}{\alphaChannelw}{\roundedIndexw}{\xw-\xxw-\xxw}{\yw-\xxw-\xxw};


    % 定义箭头指向的坐标

    \coordinate (#8_left) at  (\xw-\xxw-\xxw-2*\yLen, \yw-\xxw+\hw/2);
    \coordinate (#8_right) at   (\xw+\hww+\yLen+\yLen, \yw-\xxw+\hw/2);
    \coordinate (#8_top) at    (\xw-\xxw+\hww/2, \yw+\hww+\yLen);
    \coordinate (#8_bottom) at (\xw-\xxw+\hww/2, \yw-\xxw-\xxw-\yLen);

    % \node at (#8_left) {left};
    % \node at (#8_right) {left};
    % \node at (#8_top) {top};
    % \node at (#8_bottom) {bottom};

}
