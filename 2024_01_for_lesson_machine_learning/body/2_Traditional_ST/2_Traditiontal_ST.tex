\section{传统风格迁移技术}

在神经网络兴起、风格迁移出现以前,就有类似的技术实现对艺术图像的模拟,非真实感绘制(Non-Photorealistic Rendering,NPR)与纹理模拟(Stroke-Based Rendering)是其中两种被广泛研究的技术。

%这一段是抄的,如果要发表需要重新组织
自20世纪90年代中期以来,艺术作品背后的艺术理论不仅吸引了艺术家,也吸引了许多计算机科学研究人员的关注。有大量研究和技术探索如何将图像自动转换为合成艺术品。在这些研究中,非真实感绘制(NPR)的取得了较大进展,如今它已成为计算机图形学界一个牢固确立的领域。然而,大多数 NPR 风格化算法都是针对特定的艺术风格设计的,并且不能轻易扩展到其他风格。在计算机视觉领域,风格迁移通常被研究为纹理合成的广义问题,即将纹理从源提取并迁移到目标。 \cite{QianWenHuaFeiZhenShiGanHuiZhiJiShuYanJiuXianZhuangYuZhanWang2020}

%这一段是抄的,如果要发表需要重新组织
NPR技术在发展过程中,研究者从图像建模的角度出发,基于笔触渲染、图像类比、图像 滤波的方法,对水彩画、素描画、油画等大众喜闻乐 见的艺术作品,水墨画、中国书法等来自中国的艺术 作品,以及蜡染画、版画等少数民族的艺术作品进行 数字化模拟研究,产生了大量优秀的艺术风格数字 化模拟作品,并成功应用于动画、遗产保护等领域。NPR 可对特定艺术风格,如水彩画、油墨画以及中国风的古 典画等各种画风进行数字化模拟。根据渲染方式不同, NPR 可细分为三类:笔触渲染、图像类比和图像滤波。 Meier\cite{meierPainterlyRenderingAnimation1996}提出基于笔触渲染的画笔模型,可以模拟油画生成的过程;Hertzmann 等人\cite{hertzmannImageAnalogies2023}提出图像类比的概念,在有监督的状态下改变原图风格;Winnemöller等人\cite{winnemollerRealtimeVideoAbstraction2006}引入双边滤波器和高斯差分滤波器来自动生成卡通风格图像。 纹理迁移技术主要用于纹理合成,即根据参考图像来对输入图像进行纹理填充,使得生成图像具有类似于样图的纹理风格,适用于处理纹理简单且重复的图像, 如木纹、砖块和墙面等。Efros 和 Leung\cite{efrosTextureSynthesisNonparametric1999}采用马尔科夫随机场(Markov Random Field,MRF)模型,选取与待填充像素点的邻域最接近的纹理片段来对该点进行像 素填充。这是早期纹理合成的经典算法,但该方法每填充一个像素值就需要遍历一次纹理片段,时间成本很高。\cite{TangRenWeiShenJingFengGeQianYiMoXingZongShu2021}

上述传统的风格迁移具有相同的缺陷,如仅考虑了图像的低层语义信息,而忽略了其高级语义信息、生成的纹理变化较少等。