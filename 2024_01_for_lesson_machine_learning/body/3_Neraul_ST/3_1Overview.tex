\section{神经风格迁移技术}

随着神经网络的兴起,基于神经网络的风格迁移技术成为了该方向中较为主流的方法,本文称这种基于神经网络与深度学习的风格迁移技术为“神经风格迁移技术”。
为了能更好的介绍神经风格迁移技术,本文按迭代对象将神经风格迁移技术分为两类:基于像素迭代的风格迁移技术与基于模型迭代的风格迁移技术。前者的主要思想是利用损失函数对一张噪声图片进行迭代,从而得到一张风格化后的图像;后者对神经网络进行迭代,将风格图像的风格信息保存在神经网络的参数中,使该网络具有对特定的一种或多种风格进行迁移的能力。

以上两种迭代方法均有其子分类,且分类标准不同。对于“基于像素迭代的风格迁移技术”而言,可以根据使用的损失函数的类型进一步将其分类两类:基于格拉姆(Gram)矩阵的风格迁移技术与基于最大均值差(Maximum Mean Discrepancy,MMD)的风格迁移技术。%如图中所示〔此处需要插入图片:风格迁移技术的分类〕

对于“基于模型迭代的风格迁移技术”而言, 可以根据网络与网络所能进行转移的风格的数量进行分类,从而得到三个子类:一个风格对应一个网络模型,多个风格对应一个网络模型,任意风格对应一个网络模型。%如图中所示。〔此处需要插入图片:风格迁移技术的分类〕

本节以上述标准为主要行文思路,逐个对神经风格的代表性成果进行介绍。
