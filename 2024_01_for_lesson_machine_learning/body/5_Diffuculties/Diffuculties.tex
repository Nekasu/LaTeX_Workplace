\section{领域前沿与挑战}

风格迁移领域中的算法目前已经取得了令人惊叹的成果,但当前依旧存在一些挑战与悬而未决的问题。

\textbf{评价标准。}研究人员需要一些可靠的标准来评估他们提出的方法相对于现有技术的优势,并且还需要一种方法来评估一种特定方法对一种特定场景的适用性。然而,大多数论文通过并排主观视觉比较或通过各种用户研究得出的测量来评估他们提出的方法

\textbf{可解释性。}目前的风格迁移任务往往是基于深度学习与神经网络;同时部分成果更像是通过“发现”而非构建一个可解释的过程进行风格迁移\cite{ulyanovTextureNetworksFeedforward2016b}。这使得风格迁移的过程不可控,从而导致了结果图像可能无法满足人员的预期。

\textbf{变形问题。}目前的风格迁移算法仅考虑了在纹理与颜色上将内容图像转换为风格图像。然而部分风格画是对现实世界的抽象与简化(如动画风格与抽象派风格等),因此仅仅将纹理进行迁移是不够的。在迁移时需要对目标风格进行一定的探究,并通过设计实现在风格转换时将图像形变纳入考量。

\textbf{对纹理和颜色进行迁移。}有时,人们希望保留原始图像的颜色,仅将风格图像的纹理迁移至原始的内容图像上,然而当前算法往往同时将纹理与颜色同时迁移至内容图像上。因此仅对图像或仅对纹理进行迁移也是当前需要解决的问题之一。

\textbf{人机交互。}目前风格迁移的发展多在考虑使用单一模型实现任意风格迁移,但实现任意风格迁移并不意味着可以利用该模型进行生产活动。在生产过程中,往往需要根据需求进行定制化的过程,所以对能够对生成过程加以干涉,从而实现根据目标生成对应风格的图像是一个很重要的问题。