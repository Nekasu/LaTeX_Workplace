\section{Method}

\subsection{Background: Alpha 通道在风格迁移任务中的缺失}

\paragraph{RGB vs. RGBA: 不透明度信息的重要性}
% 引言
在计算机视觉领域, 图像往往以 R、G、B 三个通道的格式存取与处理, 因为这已足以适应绝大多数任务场景. 然而, 当涉及到对图像中不同区域进行精细的显示与隐藏控制, 或者需要调节多张图像间的遮挡与融合程度时, 传统的 RGB 图像便显得力不从心. 
% alpha通道是什么
RGBA 图像在原有的 RGB 图像基础上增加了一个 A 通道, 即 alpha 通道, 该通道用以表示图像每个像素的不透明度. alpha 通道的概念最早可追溯至 Porter 和 Duff 在 1984 年提出的数字图像合成理论, 该理论系统阐述了如何利用 alpha 通道实现图层间的混合与透明度控制\cite{porter1984compositing}. 在实际应用中, alpha 值通常用来指示像素的不透明程度, 其取值范围可以设定为 $[$0, 1$]$ 或 $[$0, 255$]$, 通过直接指定或计算生成, 以反映图像中各部分的显示或隐藏程度. 

在实际应用中, alpha通道 对于实现图像中的局部区域控制具有显著优势. 以图形设计与合成为例, alpha通道 允许设计师指定图像的前景与背景, 并精确调控多图像之间的遮挡关系. 相比之下, 传统的 RGB 图像由于仅包含颜色信息, 在需要实现精细区域控制时往往显得不足.
设想有两幅图像:一幅前景图像为人物肖像, 以 PNG 格式(一种常见的RGBA格式)存储, 其中人物区域完全不透明 (alpha 值接近 1 或 255) , 而背景部分则完全透明 (alpha 值接近 0) ; 另一幅图像为背景图像, 例如一幅风景图. 在传统的 RGB 图像处理中, 仅保留颜色信息的前景图像无法明确区分出真正的前景和需要由背景填充的区域, 因此直接将 RGB 前景叠加到背景上往往会产生不自然的矩形边框和锐利的边缘. 通过提取 PNG 图像中的 alpha 通道, 并将其作为蒙版来使用, 可以仅保留前景图像中不透明区域, 将透明区域交由背景图显示, 从而达到前景与背景自然过渡的效果. 这种方法不仅保留了前景对象的完整形态, 还使背景能够无缝融入, 为图像合成提供了更高的精确性与美观度. 

当前风格迁移工作\cite{johnson2016perceptual,risser2017stable,sanakoyeu2018style,jing2019neural,goodfellow2020generative,li2023frequency,fu2023neural,tang2022few,kwon2024aesfa}大多忽视了图形设计与合成的应用场景. 在网络输入端, 这些方法通常仅接受3通道的 RGB 图像作为输入, 而当输入为 RGBA 图像时, 系统只处理 RGB 部分, 忽略了其中的 alpha 通道. 在网络的处理中, 由于随意抛弃 alpha 信息, 导致生成过程中处理的图像与原始 RGBA 图像之间产生较大差距. 网络可能将这些差距视为图像风格的一部分, 进而在输出图像中引入大量不属于原输入图像的风格信息, 不仅使风格迁移的质量大幅下降, 也无法保留原图的不透明度信息. 这对于实际应用场景——如需要精细控制前景与背景, 以实现无缝图层合成的图形设计工作来说, 是致命的缺陷. 

\paragraph{alpha通道与 vanilla convolution 的不适配性}

RGBA 图像中的 alpha 通道与 RGB 三个通道在信息表达、语义含义及处理需求上存在本质差异. RGB 通道主要承载图像的颜色与亮度信息, 直接参与图像的视觉重建, 而 alpha 通道则用于编码像素的不透明度, 决定图像在合成、遮挡与可视化中的显隐关系, 更多体现为一种控制性的"元信息". 与 RGB 通道的结构性视觉特征不同, alpha 通道的数值变化通常呈现出遮挡块状、边界过渡或局部渐变等规律, 具有独立的数据分布与语义特征. 

然而, 当前工作常使用 vanilla convolution 统一处理输入通道. 作为一个通用模块, vanilla convolution 无法区分 RGB 通道与 alpha通道 之间的本质差异. 因此在训练与推理过程中, vanilla convolution 会错误的将 alpha通道 当作 颜色通道处理, 导致语义污染, 最终导致风格迁移质量下降. 因此, 在处理 RGBA 图像时, 有必要引入对 alpha 通道敏感的建模机制, 以避免信息在特征提取过程中的语义丢失或功能混淆.


\subsection{Soft Partial Convolution for Countinuous Alpha Transparency}

引入 alpha通道敏感的建模机制有两种实现方式. 其一, 为alpha通道 与 RGB 通道分别设计 Encoder, 并在二者之间实现信息交流; 其二, alpha 信息看作 RGB 的补充信息, 用融合策略将 alpha 通道融入主编码网络中, 在统一的特征空间中捕捉颜色与透明度的细粒度特征. 考虑到 alpha通道 用于控制像素显隐程度, 在风格迁移任务中可以表示某区域的风格化程度, 可以作为主网络的补充信息; 同时考虑到计算成本, 故本文以第二种实现方式作为主要研究对象.

为了将代表风格化程度的 alpha 信息融入主网络, 我们参考了 Image Inpainting 任务的 Partial Convolution\cite{liu2018image}. Partial Convolution 的核心思想是利用一个二值掩膜来区分已知和缺失区域, 从而仅对已知区域进行卷积运算. 然而, 对与具有连续 alpha 通道信息的 RGBA 图像而言, alpha 通道的像素值通常在 $[$0, 255$]$ 之间变化, 用于反应像素的不透明度与风格化程度. 在这种情况下, 简单地将 alpha 通道二值化容易丢失细腻的透明度信息, 从而影响网络对图像透明边缘与分层合成的精细建模. 因此, 为了让部分卷积更好适配风格迁移任务, 我们有必要对部分卷积进行修改, 使其能够直接利用连续值作为掩膜, 从而更好地保留 alpha 通道的语义信息.

我们提出了 Soft Partial Convolution for image style transfer network. 具体来说, 我们将图像中 alpha 通道归一化为 $[$0,1$]$ 之间的 soft mask, 并用该 soft mask 作为风格化程度信息, 辅助风格迁移. 假设输入特征图为 $X \in \mathbb{R}^{C\times H \times W}$, 其对应的 soft mask 由 alpha 通道归一化得到, 记做 $M_{s}\in [\mathrm{0},\mathrm{1}]$; 给定卷积核 $W \in \mathbb{R}^{C\times K \times K}$ 以及偏置 $b$, 则在 $(i,j)$ 处的 Soft Partial Convolution 输出 $Y_{i,j}$ 定义为:
\begin{equation}
    \begin{aligned}
        \label{equation-soft_partial_convolution}
        Y_{i,j} = \begin{cases}
            \frac{1}{\sum M_s^{(i+k,j+l)}} \sum_{k,l} W \cdot X^{(i+k, j+l)} \cdot M_{s}^{(i+k, j+)} + b,& \mathrm{if} \sum_{k,l} M_{s}^{(i+k, j+l)} > 0 \\
            0, & \mathrm{otherwise}
        \end{cases}
    \end{aligned}
\end{equation}
同时, 我们采样局部平均值来更新掩膜, 得到新的连续掩膜 $M_{\mathrm{out}}$
\begin{equation}
    \label{equation-mask_update}
    M_{\mathbf{out}}^{(i,j)} = \frac{1}{K^2}\sum_{k,l} M_{s}^{(i+k, j+l)} 
\end{equation}

值得一提的是, Gated Convolution\cite{yu2019free} 通常被认为是对 Partial Convolution 的改进, 其核心创新在于将掩膜的更新策略从基于规则的更新机制替换为可学习的门控机制. 该机制能够在训练中自动学习图像中哪些区域应当被保留, 从而在某些任务中(如图像修复)取得了更灵活的表现. 然而, 在本研究聚焦的风格迁移任务中, 我们发现 Partial Convolution 具备更强的结构适配性. 首先, Gated Convolution 更适合目标区域不确定的情况, 而在风格迁移中, 掩膜(即 alpha 通道)来源于输入图像本身, 具有明确且一致的语义信息, 无须使用额外的网络进行门控预测. 其次, 风格图像与内容图像的 alpha 通道提供了明确的不透明度信息, 我们更喜欢模型严格遵循这一先验知识建模, 而非通过门控机制学习透明度信息, 以避免不必要的偏差. 虽然 Gated Convolution 作为 Partial Convolution 的改进在某些任务中取得了优越表现, 但在本研究面向的 alpha 感知风格迁移任务中,  Partial Convolution 由于其简洁与精确的特性, 表现出了更高的适用性.

\subsection{Style Transfer Network Architecture}

我们使用所提出的 Soft Partial Convolution 与 AesFA 网络\cite{kwon2024aesfa} 定制了一个能够处理 RGBA 图像的风格迁移网络. 具体来说, 我们采用并修改了 \cite{kwon2024aesfa} 中的主干网络结构, 在保持原始网络优势的基础上, 额外引入了预处理模块与后处理模块, 以更好地适应带有透明度信息的图像建模需求. 

在输入 RGBA 图像时, 原始网络的 Encoder 会对所有通道进行统一的特征提取, 未对 alpha 通道与 RGB 通道加以区分. 为解决这一缺陷, 我们在 Encoder 之前增加了预处理模块, 使得 RGB 通道与 alpha 通道被明确分离, 并分别建模. 其中, RGB 图像经过 Soft Partial Convolution 处理, 而 alpha 通道被视为 soft mask, 在整个建模过程中作为引导信息持续参与. 若输入图像本身不含 alpha 通道(例如 JPG 格式图像), 该模块会自动补充一个值均为 255 的 alpha 通道, 以增强网络在多图像格式下的鲁棒性与适应性. 

在 Decoder 输出图像后, 我们进一步引入了后处理模块. 该模块以原始输入图像的 alpha 通道为参考, 对生成图像的有效区域进行裁剪, 使网络输出不仅保持风格一致性, 还能严格遵守原图像的透明度约束. 这一机制对于图形合成与界面设计等应用场景具有重要意义, 确保风格迁移结果能无缝集成于复杂图形层结构中. 

此外, 我们将原始网络\cite{kwon2024aesfa}中的所有 vanilla convolution 替换为 Soft Partial Convolution, 以更好地融合透明度信息带来的结构引导. 在此基础上, 我们适当加深了网络结构的整体深度, 并将特征提取阶段原本使用的双尺度特征图处理机制拓展为三尺度设计, 从而增强网络的风格表达能力与多层次语义建模能力. 值得一提的是, 我们的网络仍保持全卷积结构, 能够灵活适配任意分辨率的输入图像, 进一步提升其实用性与泛化性.  整体网络结果如图所示(picture needed).