\section{Traditional Style Transfer}

The term "style transfer" is usually used to refer to the second phase of neural style transfer that began after the publication of Gatys et al.'s paper in 2016 \citep{02gatys2016image}. Before this, the term "style transfer" was not widely accepted. The work in the first phase is often referred to as Non-Photorealistic Rendering (NPR) or Image-Based Artistic Rendering (IB-AR).

In this paper, we refer to the work from the first phase (1990s-2016) as traditional style transfer to help readers better understand the continuity between the two approaches. We follow the suggestion from \citep{01jing2019neural} and adopt the classification method for traditional style transfer proposed in \citep{21kyprianidis2012state}. However, the focus of this paper is not on traditional style transfer, especially since traditional methods have now been largely integrated with neural style transfer techniques. Therefore, we recommend readers who require a systematic understanding of traditional style transfer methods to refer to these two works\citep{01jing2019neural,21kyprianidis2012state}.

When introducing the achievements of traditional style transfer, this paper will proceed by discussing their characteristics, advantages, and limitations.

\textbf{Stroke-Based Rendering (SBR)}is a core algorithm in traditional style transfer that involves covering a 2D canvas with atomic-level rendering primitives to simulate specific artistic styles. These primitives typically include virtual brushstrokes, patches, stippling, and shading marks.

The most common form of SBR is rendering with virtual brushstrokes. The color, direction, size, and order of these strokes may be determined semi-automatically or automatically. The stylized output depends not only on the simulation of the medium used to render each stroke but also on the process of stroke placement and the methods used to set their attributes. The stroke placement process in SBR can be roughly divided into local and global approaches. Local methods typically base stroke placement decisions on the pixels within the spatial neighborhood of the stroke. This can be explicitly specified in the algorithm (e.g., image moments within a window) or implied by previous convolution operations (e.g., Sobel edges). A branch of SBR uses media other than colored pixels or paint to fill image regions, including using small dots (stippling) for tone description, line patterns or curves (shading marks), and mosaic algorithms that combine small tiles. When handling video content, the motion of the strokes should match the movement in the video content. This is particularly emphasized in SBR algorithms to ensure dynamic effects and visual coherence in videos. The advantage of SBR lies in its ability to create effects that closely resemble traditional artworks, making it especially suitable for mimicking styles such as oil painting, watercolor, and sketching. However, SBR methods may struggle when dealing with highly complex or abstract artistic styles, as these styles may not be easily achieved through traditional stroke simulation.


\textbf{Example-Based Rendering (EBR)}is a technique aimed at learning and mimicking specific artistic styles. It does so by analyzing the mapping relationship between an example pair (e.g., an original image and an artist's rendered version of that image) and then applying this mapping to stylize other images. Such methods typically encode a set of heuristic rules to faithfully depict the intended style. They try to capture the essential characteristics of a style by learning and imitating the techniques and styles an artist applied in a particular work. Once the mapping is learned, it can be used to stylize arbitrary images, making them visually similar to the original example images. This approach not only mimics specific artistic styles but can also replicate the unique style of a particular artist. The advantage of EBR is that it can produce highly personalized results, making it particularly suitable for imitating the style of specific artists or works. However, this method depends heavily on high-quality example pairs, and it may be difficult to achieve the desired stylization effect if suitable training data is lacking.

\textbf{Image Processing and Filtering (IPF)}is based style transfer uses various image processing filters and algorithms to achieve artistic stylization. This includes techniques that are based on image pyramids, as well as the use of interactive techniques (such as human gaze trackers and saliency maps) to explore different levels of an image. Various filtering techniques explore different image processing filters used for artistic stylization, but to date, only a few results have been considered interesting from an artistic perspective. This may be because these filters usually focus on the restoration and enhancement of photorealistic images. The image pyramid and interactive techniques (such as human gaze trackers and saliency maps) explore hierarchical representations by segmenting different resolution versions of the source image. High-level abstraction is achieved by rendering only the coarse large regions or specific areas located at the top of the pyramid. This method helps capture larger shapes and compositional features in the image, rather than detailed textures or lines. Unlike the simplification typically pursued in Image-Based Artistic Rendering (IB-AR), these filtering methods are often associated with the restoration and enhancement of photorealistic images. The advantage of this approach lies in its ability to quickly and easily apply stylization to images, making it suitable for creating diverse and abstract visual effects. However, such methods lack the fine detail and complexity required for artistic stylization, making it difficult to precisely mimic the subtle features of specific artists or styles.

\textbf{Summary}\quad As pioneers in the field of style transfer, these methods have inspired the emergence and development of neural style transfer, but they share some common shortcomings. Each different style transfer method embodies the author's understanding of a particular style. However, because these methods often incorporate the author's interpretation of the style, the transfer results may be correlated with the author's aesthetic level, leading to varying quality in the stylized output. Additionally, the algorithms designed for the same or similar textures or styles are often similar or even identical, resulting in stylized images with textures that may appear rigid and dull. The limited generalization capability also restricts the broad application of traditional style transfer methods. Traditional methods are designed for specific types of styles and images, and their effectiveness diminishes when generalized to different styles or images.