%% Use \section commands to start a section
\section{Example Section}
\label{sec1}
%% Labels are used to cross-reference an item using \ref command.

Section text. See Subsection \ref{subsec1}.

%% Use \subsection commands to start a subsection.
\subsection{Example Subsection}
\label{subsec1}

Subsection text.

%% Use \subsubsection, \paragraph, \subparagraph commands to 
%% start 3rd, 4th and 5th level sections.
%% Refer following link for more details.
%% https://en.wikibooks.org/wiki/LaTeX/Document_Structure#Sectioning_commands

\subsubsection{Mathematics}
%% Inline mathematics is tagged between $ symbols.
This is an example for the symbol $\alpha$ tagged as inline mathematics.

%% Displayed equations can be tagged using various environments. 
%% Single line equations can be tagged using the equation environment.
\begin{equation}
f(x) = (x+a)(x+b)
\end{equation}

%% Unnumbered equations are tagged using starred versions of the environment.
%% amsmath package needs to be loaded for the starred version of equation environment.
\begin{equation*}
f(x) = (x+a)(x+b)
\end{equation*}

%% align or eqnarray environments can be used for multi line equations.
%% & is used to mark alignment points in equations.
%% \\ is used to end a row in a multiline equation.
\begin{align}
 f(x) &= (x+a)(x+b) \\
      &= x^2 + (a+b)x + ab
\end{align}

\begin{eqnarray}
 f(x) &=& (x+a)(x+b) \nonumber\\ %% If equation numbering is not needed for a row use \nonumber.
      &=& x^2 + (a+b)x + ab
\end{eqnarray}

%% Unnumbered versions of align and eqnarray
\begin{align*}
 f(x) &= (x+a)(x+b) \\
      &= x^2 + (a+b)x + ab
\end{align*}

\begin{eqnarray*}
 f(x)&=& (x+a)(x+b) \\
     &=& x^2 + (a+b)x + ab
\end{eqnarray*}

%% Refer following link for more details.
%% https://en.wikibooks.org/wiki/LaTeX/Mathematics
%% https://en.wikibooks.org/wiki/LaTeX/Advanced_Mathematics

%% Use a table environment to create tables.
%% Refer following link for more details.
%% https://en.wikibooks.org/wiki/LaTeX/Tables
\begin{table}[t]%% placement specifier
%% Use tabular environment to tag the tabular data.
%% https://en.wikibooks.org/wiki/LaTeX/Tables#The_tabular_environment
\centering%% For centre alignment of tabular.
\begin{tabular}{l c r}%% Table column specifiers
%% Tabular cells are separated by &
  1 & 2 & 3 \\ %% A tabular row ends with \\
  4 & 5 & 6 \\
  7 & 8 & 9 \\
\end{tabular}
%% Use \caption command for table caption and label.
\caption{Table Caption}\label{fig1}
\end{table}


%% Use figure environment to create figures
%% Refer following link for more details.
%% https://en.wikibooks.org/wiki/LaTeX/Floats,_Figures_and_Captions
\begin{figure}[t]%% placement specifier
%% Use \includegraphics command to insert graphic files. Place graphics files in 
%% working directory.
\centering%% For centre alignment of image.
\includegraphics{example-image-a}
%% Use \caption command for figure caption and label.
\caption{Figure Caption}\label{fig2}
%% https://en.wikibooks.org/wiki/LaTeX/Importing_Graphics#Importing_external_graphics
\end{figure}


%% The Appendices part is started with the command \appendix;
%% appendix sections are then done as normal sections