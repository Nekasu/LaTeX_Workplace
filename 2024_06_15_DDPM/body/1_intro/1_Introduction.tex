\section{前言}

近年来,随着人工智能和深度学习技术的迅猛发展,图像生成模型在计算机视觉领域中发挥着越来越重要的作用。生成对抗网络(GAN)、变分自编码器(VAE)以及『基于扩散的图像生成模型』等,不仅在学术研究中取得了显著的进展,也在实际应用中展现出巨大的潜力。其中,『基于扩散的图像生成模型』通过模拟物理扩散过程,实现了高质量图像的生成和重建,成为当前图像生成研究的热点之一。

扩散过程和逆扩散过程是『基于扩散的图像生成模型』的核心机制。扩散过程模拟从清晰图像到噪声图像的逐步退化,而逆扩散过程则反其道而行之,通过一系列渐进的步骤,将噪声图像恢复为清晰图像。这一双向过程不仅为图像生成提供了新的思路,也在理论上为图像处理任务提供了新的视角。

本文旨在对扩散与逆扩散过程的理论基础和实际应用进行全面综述与深入解析。文章首先介绍了『基于扩散的图像生成模型』的基本工作流程,旨在为读者提供该类模型生成过程的概念框架。然后,详细阐述了『扩散过程与逆扩散过程』的原理及其在图像生成中的重要性。最后,通过数学公式推导,深入解析了这两个过程的具体实现,提供了系统性的理论支持。

以下是作者的一些碎碎念。

许多人可能认为DDPM是一个难以理解的过程,同时其推导也需要大量的数学知识,但其实不然。经过本人的理解,DDPM涉及的扩散过程与逆扩散过程并非难以理解,整体过程简洁而有效。DDPM的推导也并不需要过多的数学知识,推导过程中使用的最多的公式为贝叶斯公式,最困难的知识为KL散度的定义与极大似然估计的原理,所以DDPM并不是一个难以理解的知识。只要您仔细阅读并积极思考与理解,就能轻松获得关于DDPM的基础原理。为了能够获得愉快的阅读体验,本文使用口语化的表达方式;同时为了防止文字出现歧义,使用了尽可能明确的表达方式。通过以上两种方式,力求为读者提供愉悦体验的同时学到清晰明确的结论。

此外,本文以Denoising Diffusion Porbabilistic Models(DDPM)
\cite{hoDenoisingDiffusionProbabilistic2020a}及其原理解释论文\cite{luoUnderstandingDiffusionModels2022}为基础,吸收了李宏毅老师关于DDPM的机器学习课程
\cite{XiangBuChuLaiNiChengYouXiangGaiShengChengShiAIDiffusionModel2023}
\cite{XiangBuChuLaiNiChengYouXiangGaiShengChengShiAIDiffusionModel2023a}\cite{XiangBuChuLaiNiChengYouXiangGaiShengChengShiAIDiffusionModel2023b}
\cite{XiangBuChuLaiNiChengYouXiangGaiShengChengShiAIDiffusionModel2023c}
\cite{XiangBuChuLaiNiChengYouXiangGaiShengChengShiAIDiffusionModel2023d}
\cite{XiangBuChuLaiNiChengYouXiangGaiShengChengShiAIDiffusionModel2023e}
的讲解思路,参考多篇网络博客\cite{DiffusionModelsShengChengKuoSanMoXing},结合自身理解与公式推导,完成了这篇关于DDPM的直观与深入解释文章。尽管与原始论文\cite{hoDenoisingDiffusionProbabilistic2020a}在推导结果上存在的一定的差异,但是经过一定的变换后,可以得到相同的结果,且个人认为本人的推导可能在帮助理解方面能有更好的效果,所以请各位读者放心阅读。

本人花费约2个星期将DDPM的原理基本吸收,并将自己的思路与理解记录在此处,并以「力求能教会所有学过概率论的读者」的标准撰写该文档,希望本文能为图像生成领域的研究者提供有价值的参考和指导,促进该领域的进一步发展和创新。如果在阅读时遇到了名词混淆的情况,请阅读\ref{sec_noun}名词辨析作为参考。