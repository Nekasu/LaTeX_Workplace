\maketitleofchinese{基于扩散的图像生成模型与DDPM原理}{周肖桐\footnote{E-mail:\url{xiaotongzhounuist@163.com}}}{计算机学院、网络空间安全}

\abstractofchinese{本文旨在全面概述和深入分析扩散与逆扩散过程在图像生成模型中的应用与理论基础。文章分为三个主要部分:首先总览扩散模型的图像生成过程, 并作为扩散与逆扩散过程的引入。其次,详细阐述扩散过程与逆扩散过程的概念与区别,讨论这些过程在图像生成中的重要性及其工作原理。最后,通过数学公式推导,深入解析扩散与逆扩散过程的具体实现,提供系统性的理论支撑,并对其性能和有效性进行评估。本文通过对这两种过程的全面分析与推导,旨在为图像生成领域的研究者提供有价值的参考与指导。}{扩散模型、原理介绍、公式推导}


\maketitleofenglish{Diffusion Based Generation Model \& Principles of DDPM}{Xiaotong Zhou\footnote{E-mail: \url{xiaotongzhounuist@163.com}}}{School of Computer Science}

\abstractofenglish{This paper aims to provide a comprehensive overview and in-depth analysis of the application and theoretical foundation of diffusion and reverse diffusion processes in image generation models. The article is divided into three main sections: First, it introduces the basic image generating process of DDPM, aiming to explore the structural similarities of these models and to serve as an introduction to diffusion and its reverse processes. Second, it elaborates on the concepts and differences between the diffusion process and the reverse diffusion process, discussing their importance and working principles in image generation. Finally, through mathematical formula derivation, it delves into the specific implementation of diffusion and reverse diffusion processes, providing systematic theoretical support and evaluating their performance and effectiveness. By providing a comprehensive analysis and derivation of these two processes, this paper aims to offer valuable reference and guidance for researchers in the field of image generation.}
{Diffusion models, Principles introduction, Formula derivation}
